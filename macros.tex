\newenvironment{titledslide}[1]{\flushleft\begin{slide}\titolo{#1}\vfill}{\vspace{\fill}\end{slide}}
\newcommand{\titolo}[1]{\begin{center}#1\end{center}
 \vspace{-8mm} 
 \rule[10mm]{\textwidth}{0.7mm}}
\newcommand{\epspic}[2]{\epsfxsize=#2mm\begin{center}\leavevmode\epsfbox{#1.ps}\end{center}}
\newcommand{\pl}{{\tt PLaSM}} % PLaSM
\newcommand{\simplexn}{{Simple$_{X}^{n}$}} % Simplexn
\newcommand{\fl}{{\tt FL}} % FL

\newenvironment{code}
{\samepage\small\begin{tabbing}aa\=aa\=aa\=aa\=aa\=aa\=aa\=aa\=aa\=aa\=aa\=aa\=aa\=aa\kill}{\end{tabbing}}

\newenvironment{plasm}
{\samepage\tt\begin{quote}
\begin{tabbing}aa\=aa\=aa\=aa\=aa\=aa\=aa\=aa\=aa\=aa\=aa\=aa\=aa\=aa\kill}
{\end{tabbing}\end{quote}}

\newenvironment{smallplasm}
{\begin{quote}\samepage\small\tt
\begin{tabbing}aaa\=aaa\=aaa\=aaa\=aaa\=aaa\=aaa\=aaa\=aaa\=aaa\=aaa\=aaa\=aaa\=aaa\kill}
{\end{tabbing}\end{quote}}

\newenvironment{tinyplasm}
{\scriptsize\tt
\begin{tabbing}aaa\=aaa\=aaa\=aaa\=aaa\=aaa\=aaa\=aaa\=aaa\=aaa\=aaa\=aaa\=aaa\=aaa\kill}
{\end{tabbing}}

\newenvironment{smallerplasm}
{\footnotesize\tt\begin{tabbing}aaa\=aaa\=aaa\=aaa\=aaa\=aaa\=aaa\=aaa\=aaa\=aaa\=aaa\=aaa\=aaa\=aaa\kill}{\end{tabbing}}

\newenvironment{smallestplasm}
{\tt\tiny\begin{tabbing}aaa\=aaa\=aaa\=aaa\=aaa\=aaa\=aaa\=aaa\=aaa\=aaa\=aaa\=aaa\=aaa\=aaa\kill}{\end{tabbing}}

\newenvironment{plasmverb}
{\small\tt\begin{verbatim}}{\end{verbatim}}

% \newenvironment{script}
% {\noindent\rule{5.15in}{0.1mm}\vspace{-0.2cm}\begin{plist}}
% {\end{plist}\vspace{-0.4cm}\rule{5.15in}{0.1mm}\vspace{0.2cm}}

\newenvironment{defscript}
{\begin{minipage}[c]{\linewidth}\begin{plist}}{\end{plist}\end{minipage}}

%\newcommand{\ve}[1]{\bf #1}
\def\E{{\sf I\!E}}

\def\D#1#2{{d^{#1} \over d #2}}
\def\Partial#1#2{{\partial^{#1} \over \partial #2}}
\def\v#1{{\bf #1}}
\def\v#1{{\mbox{\boldmath $#1$}}}
\def\p#1{{\bf #1}}
\def\p#1{{\mbox{\boldmath $#1$}}}
\def\T#1{{\bf #1}}
\def\T#1{{\mbox{\boldmath $#1$}}}
\def\grad#1{\mbox{grad\,}#1}
%\def\vect2 #1 #2{{\left[\begin{array}{cc}#1\\#2\end{array}\right]}}
\def\vet#1{{\left(\begin{array}{ccccccc}#1\end{array}\right)}}
\def\mat#1{{\left(\begin{array}{ccccccc}#1\end{array}\right)}}
\def\deriv#1#2{{d #1 \over d #2}}
\def\pderiv #1 #2{{ \partial #1} \over {\partial #2}}
\def\binomial#1#2{{\left(\!\begin{array}{c}#1\\#2\end{array}\!\right)}}
\def\module#1{{\left|\!\left|#1\right|\!\right|}}


\newtheorem{definition}{Definition}[section]
\newtheorem{Note}{Note}[section]
\newtheorem{rrule}{Rule}[section]
\newtheorem{lemma}{Lemma}[section]
\newtheorem{theorem}{Theorem}[section]
\newtheorem{corollary}{Corollary}[section]

\theorembodyfont{\rmfamily}
\theoremstyle{break}
\newtheorem{example}{Example}[section]
\newtheorem{plist}{Script}[section]

\def\E{\mathbb{E}}
\def\R{\mathbb{R}}
\def\P{\mathbb{P}}
\def\N{\mathbb{N}}
\def\Z{\mathbb{Z}}
\newcommand{\curl}[0]{\mbox{\rm curl}\,}
\newcommand{\topint}[0]{\mbox{\rm int}\,}
\newcommand{\closure}[0]{\mbox{\rm clos}\,}
\newcommand{\lin}[0]{\mbox{\rm lin}\,}
\newcommand{\aff}[0]{\mbox{\rm aff}\,}
\newcommand{\pos}[0]{\mbox{\rm pos}\,}
\newcommand{\cone}[0]{\mbox{\rm cone}\,}
\newcommand{\conv}[0]{\mbox{\rm conv}\,}
\newcommand{\relint}[0]{\mbox{\rm relint}\,}
\newcommand{\homog}[0]{\mbox{\rm homog}\,}
\newcommand{\ext}[0]{\mbox{\rm ext}\,}
\newcommand{\complement}[0]{\mbox{\rm complement}\,}
\newcommand{\proj}[1]{\mbox{\rm proj}_{#1}\,}
\newcommand{\elim}[1]{\mbox{\rm elim}_{#1}\,}
% \newcommand{\dim}[0]{\mbox{\rm dim}\,}


\def\myDelta{{\scriptscriptstyle{\Delta}}}
\def\one{{\rm \mbox{\small $1$}\!\!1}}

%-----------------------------------------------------------------------------
% \newcommand{\figsdir}[1]{os9:Work:Book:#1:#1fig:}
% \newcommand{\exampledir}[1]{os9:Work:Book:examples:#1:}
\newcommand{\figsdir}[1]{/Users/paoluzzi/BACK/Users/work/Libri/book/#1/}
%-----------------------------------------------------------------------------


%%% Claudio Baldazzi

\newcommand{\bigsymb}[3]{\begin{array}{c}
{\scriptscriptstyle #3}\\[-0.15cm]
{\displaystyle #1} \\[-0.3cm]
{\scriptscriptstyle #2}\end{array}}

\newcommand{\BSP}{\mbox{BSP}}

\newenvironment{lang}
{
\begin{quote}
    \begin{tabbing}
        aaa\=aaa\=aaa\=aaa\=aaa\=aaa\kill
}
{
\end{tabbing}
\end{quote}
}

\newcommand{\note}[1]{\typeout{--------Note: #1}{\sc #1}}
\newcommand{\st}[0]{\;|\;}                           %tale che

% background section
\newcommand{\espace}[0]{E} 
\newcommand{\rspace}[0]{\R}
\newcommand{\vset}[0]{V}
\newcommand{\vsymb}[0]{v}
\newcommand{\fsymb}[0]{f}
\newcommand{\dime}[0]{\mbox{\rm dim}\,}

\newcommand{\cella}[0]{\cell}
\newcommand{\interior}[1]{\stackrel{\circ}{#1}}
\newcommand{\boundary}[0]{\partial}
\newcommand{\interiorcella}[0]{\interior{\cella}}
\newcommand{\insieme}[0]{\rm}
\newcommand{\dcomplesso}[0]{{\cal K}}
\newcommand{\identita}[0]{I} 

\newcommand{\poliedro}[0]{P}  \newcommand{\poliedri}[0]{{\cal P}}
\newcommand{\mappa}[0]{T}     \newcommand{\mappe}[0]{{\cal T}}
\newcommand{\istanza}[0]{I}   \newcommand{\istanze}[0]{{\cal I}}
\newcommand{\complesso}[0]{C} \newcommand{\complessi}[0]{{\cal C}}
\newcommand{\modelli}[0]{{\cal M}}

\newcommand{\Rpoliedro}[0]{\langle P \rangle}
\newcommand{\Ristanza}[0]{\langle I \rangle}
\newcommand{\Rcomplesso}[0]{\langle C \rangle}
\newcommand{\Rmappa}[0]{\langle T \rangle}
\newcommand{\rappresentazioni}[0]{{\cal R}}
\newcommand{\schemaR}[0]{\mbox{\tt r}}
\newcommand{\schemaWR}[0]{\mbox{\tt wr}}

\newcommand{\istanzaB}[0]{\tilde{I}}  \newcommand{\istanzeB}[0]{\tilde{{\cal I}}}
\newcommand{\sequenza}[0]{S}          \newcommand{\sequenze}[0]{{\cal S}}
\newcommand{\Modelli}[0]{{\cal M}^*}
\newcommand{\Rappresentazioni}[0]{{\cal R}^*}
\newcommand{\RistanzaB}[0]{\langle \tilde{I} \rangle}
\newcommand{\Rsequenza}[0]{\langle S \rangle}
\newcommand{\schemaRR}[0]{\mbox{\tt r}^*}
%\newcommand{\diff}[0]{\mbox{\tt V}}
%\newcommand{\Rdiff}[0]{\mbox{\tt V}^*}
\newcommand{\diff}[0]{\mbox{\tt PD}}
\newcommand{\Rdiff}[0]{\mbox{\tt PD}^*}
\newcommand{\oper}[0]{\star}
\newcommand{\operX}[0]{\diamond_\star}
\newcommand{\Rom}[0]{\tt}
\newcommand{\operA}[0]{\operX^{\Rom 1}}
\newcommand{\operB}[0]{\operX^{\Rom 2}}
\newcommand{\operC}[0]{\operX^{\Rom 3}}
\newcommand{\operD}[0]{\operX^{\Rom 4}}

%paper operatori
%scheletro
\newcommand{\kskeleton}[0]{{\tt @k}}
\newcommand{\kskeletonX}[0]{\diamond_{\kskeleton}}
\newcommand{\kskeletonB}[0]{\kskeletonX^{\Rom 2}}
\newcommand{\cell}[0]{c}        % coerenza con  \cella !!!!!!!!!
\newcommand{\rank}[0]{\mbox{\rm rank}\,}

%prodotto
\newcommand{\prodottoX}[0]{\diamond_{\prodotto}}
\newcommand{\sequenzaB}[0]{D}
\newcommand{\poliedroA}[0]{\poliedro}
\newcommand{\poliedroB}[0]{Q}
\newcommand{\poliedroC}[0]{O}
\newcommand{\mappaI}[0]{T_{\identita}}
\newcommand{\matrici}[0]{M}
\newcommand{\prodottoB}[0]{\prodottoX^{\Rom 2}}
\newcommand{\prodottoC}[0]{\prodottoX^{\Rom 3}}
\newcommand{\facce}[0]{F}
\newcommand{\celle}[0]{\dcomplesso}
\newcommand{\cellepoliedroA}[0]{\celle_{\poliedroA}}
\newcommand{\cellepoliedroB}[0]{\celle_{\poliedroB}}
\newcommand{\cellepoliedroC}[0]{\celle_{\poliedroC}}



\newcommand{\umlaut}[1]{\"{#1}}
\newcommand{\Simplex}[0]{\emph{Simple}${}_X^n$\ }
\newcommand{\simplex}[0]{\emph{Simple}${}_X^n$\ }
\newcommand{\rem}[1] {\rule{3mm}{.2mm}\hspace{2mm}{\sc
   #1}\hspace{2mm}\rule{3mm}{.2mm}\hspace{2mm}} 
\newcommand{\prodotto}[2]  {\,\, {}_{#1} \otimes_{#2} \,\,}
\newcommand{\Prodotto}[3]  {\,\, {}_{#1}\!\!\otimes^{#3}_{#2} \,}
\newcommand{\pol}[1]  {{\cal P}^{#1}}
\newcommand{\af}[1]  {{\cal A}^{#1}}
\newcommand{\fa}[1]  {{\cal A}_{#1}}
\newcommand{\ad}[1]{{\cal AD}_{#1}}

\def\vint{\int\!\!\int\!\!\int}
\def\sint{\int\!\!\int}
\def\isum{\sum_{i=1}^n}
\def\asum{\sum_{\alpha=0}^n}
\def\bsum{\sum_{\beta=0}^m}
\def\csum{\sum_{\gamma=0}^p}
\def\sssum{\asum\bsum\csum}
\def\XXa{\sum_{\tau\in T}  |{\bf a} \times {\bf b}| }
\def\XXb{\pmatrix{\alpha\cr  h\cr}}
\def\XXc{\pmatrix{\beta\cr k\cr}}
\def\XXd{\pmatrix{\gamma\cr m\cr}}
\def\XXe{\pmatrix{h\cr i\cr}} 
\def\XXf{\pmatrix{k\cr j\cr}}
\def\XXg{\pmatrix{m\cr l\cr}}
\def\XXh{\pmatrix{p\!\!+\!\!1\cr h\cr}}

\newlength\myheight  %command for \inlinegraphics
\newlength\mydepth
\settototalheight\myheight{Xygp}
\settodepth\mydepth{Xygp}
\setlength\fboxsep{0pt}
\newcommand*\inlinegraphics[1]{%
  \settototalheight\myheight{Xygp}%
  \settodepth\mydepth{Xygp}%
  \raisebox{-\mydepth}{\includegraphics[height=\myheight]{#1}}%
}

%Command for documenting functions

% \newcommand{\function}[5]{
% }}\noindent
% \vspace{0.1cm}\rule{5.15in}{0.1mm}\vspace{-0.2cm}
% \begin{description} \item[{\tt ~#1}] #2\end{description}
% \vspace{-3mm}
% \begin{tabular}[t]{lp{4.2in}}
% \ifthenelse{\equal{#3}{}}{}{{\small Prototype:}&{\small \texttt{#3}}\\}
% \ifthenelse{\equal{#4}{}}{}{{\small Returns:}&{\small \texttt{#4}}}
% \ifthenelse{\equal{#5}{}}{}{\\ {\small Example:}&{\small \texttt{#5}}}
% \end{tabular}}

\newcommand{\function}[6]{
}\noindent
\vspace{0.1cm}\rule{5.15in}{0.1mm}\vspace{-0.4cm}
\begin{description} \item[{\tt ~#1}] {\small #3}\end{description}
\vspace{-2mm}
\begin{tabular}[t]{lp{4.2in}}
\ifthenelse{\equal{#4}{}}{}{{\small \hspace{-0.3mm}Pre/Post conds}&{\small \texttt{#4}}}
\ifthenelse{\equal{#5}{}}{}{{${} \rightarrow {}$}{\small \texttt{#5}}\\}
\ifthenelse{\equal{#6}{}}{}{ {\small \hspace{-1.4mm}Example}&{\small \texttt{#6}}}
\end{tabular}}

%Ignore overfull/underfull warnings

\hbadness=30000
\vbadness=30000
\showboxbreadth=0
\showboxdepth=0
\hfuzz=\maxdimen
\vfuzz=\maxdimen


% macros for book formatting

%\newfloat{bookcopyright}{b}{ext}{}
%\newfloat{floatplasm}{htb}{ext}{}

\newenvironment{script}
{\begin{floatplasm}\noindent\rule{\textwidth}{0.1mm}\vspace{-0.2cm}\begin{plist}}
{\end{plist}\vspace{-0.4cm}\rule{\textwidth}{0.1mm}\vspace{-0.4cm}\end{floatplasm}}

\newenvironment{nofloat}
{\noindent\rule{5.15in}{0.1mm}\vspace{-0.2cm}\begin{plist}}
{\end{plist}\vspace{-0.4cm}\rule{5.15in}{0.1mm}}

%\def\gp4cad{\begin{bookcopyright}
%    \emph{Geometric Programming for Computer-Aided Design}\ \ Alberto Paoluzzi\\
%    \copyright\ 2003 John Wiley \& Sons, Ltd\ \ ISBN 0-471-89942-9
%\end{bookcopyright}}

\makeindex                                         
\def\sysdetails{}
\linewidth = \textwidth

