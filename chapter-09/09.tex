% !TEX TS-program = xelatex

\chapter{Building Information Modeling (BIM)}
\label{chapt:9}

The construction industry is nowadays undergoing its very own digital revolution, helped by its unique way of working with BIM systems whose elements are the digital prototypes of physical elements like columns, windows, walls, doors, and stairs. 
In this chapter, we outline the conceptual foundations and the down of the BIM attitude of the construction industry and discuss the birth of concepts like building objects. We also discuss the standardized taxonomy of building and space systems as set of subsystems devoted to the fulfillment of physical requirements with suitable performance levels. 
In particular, we show  by example how Julia |Plasm|, relying on its understand of topology and constructive techniques, may automatically transform the preliminary shape design of spaces into the generic geometry of building elements and subsystems (skeleton, envelope, internal partitions, horizontal floors, vertical communications).


\section{BIM history (Chuck Eastman, ...)}\label{sect:9-1}


\section{Building taxonomy (UNI 9838)}\label{sect:9-2}


\section{Building envelope}\label{sect:9-3}


\section{Building skeleton}\label{sect:9-4}


\section{Construction Process Modeling}\label{sect:9-5}


