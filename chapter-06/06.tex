% !TEX TS-program = xelatex

\chapter{Product assembly structure}
\label{chapt:6}

Hierarchical models of assemblies are generated by aggregating subassemblies, each defined in its local coordinate system and relocated by affine transformations. This operation may be repeated hierarchically, with some subassemblies defined by aggregating simpler parts, and so on, until we get a set of elementary components that cannot be further disaggregated.
Two main advantages can be found in hierarchical modeling. Each elementary part and each assembly, at every hierarchical level, are defined independently from each other, using local coordinate frames suitably chosen to make their definition easier. Furthermore, only one copy of each component is used, and it can be instanced in different locations and orientations, depending on how many times it is needed. In this chapter, after an overview of solid modeling data structures, the models of the |Plasm| dataset are discussed. 


\section{Hierarchical ssembly definition}\label{sect:6-1}


\section{Data structures in solid modeling}\label{sect:6-2}


\section{Structure in PHIGS and Plasm}\label{sect:6-3}


\section{Julia Plasm data structures}\label{sect:6-4}


\subsection{Hierarchical Polyhedral Complex (HPC)}\label{sect:6-4-1}


\subsection{Linear Algebraic Representation (LAR)}\label{sect:6-4-2}


\subsection{Geometric DataSet (GEO)}\label{sect:7-4-3}


