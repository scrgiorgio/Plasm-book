% !TEX TS-program = xelatex

\chapter{Space arrangements}
\label{chapt:7}

The chapter introduces computational topology algorithms to discover the two-dimensional (2D)/3D space partition induced by a collection of geometric objects of dimension (1D)/2D, respectively. 
The data structures needed for such computational program are sparse arrays and their standard algebraic operations. In this chapter, we also introduce a novel approach to solid modeling based on piecewise-linear algebraic topology, that allows to treat rather general cellular complexes, with cells homeomorphic to polyhedra, i.e., to triangulable spaces, and hence possibly non-convex and multiply connected. The notions we deal with include geometric complexes, linear spaces of chains and cochains, the chain complex of linear operators between pairs of spaces, and their compositions. The discussion is restricted to piecewise-linear topology and to space dimensions less or equal to three. 

\section{ Space partition and enumeration}\label{sect:7-1}


\section{ Cellular and boundary models}\label{sect:7-2}


\section{ Arrangements and Lattices}\label{sect:7-3}


\section{ 2D and 3D Examples}\label{sect:7-4}


