% !TEX TS-program = xelatex

\chapter{Boolean solid algebras}
\label{chapt:8}

Engineers conceived Solid Modeling as a rigorous and universal language for geometry-based engineering. Mathematically, all solid models are computer representations of elements of some algebraic systems that are constructed and maintained by algorithms corresponding to the operations in such algebra.
This chapter, in particular, focuses on the algebraic relationships between CSG (Constructive Solid Geometry) and boundary representations of piecewise-linear (PL) polyhedra. We show that the (regularized) arrangement of a given set of spatially instanced primitive shapes is isomorphic to the finite Boolean algebra of regularized sets containing all possible CSG representations with the same primitives.
In particular, we show that the boundary evaluation of any CSG representation with a finite number of primitives, once executed some point membership tests, reduces algebraically to the Julia form of Boolean operations on bit strings, sparse matrix-vector multiplication, and shape reconstruction from algebraic atoms. Along the way, we discuss eloquent examples.

 

\section{Constructive Solid Geometry (CSG)}\label{sect:8-1}


\section{Atoms and Generators }\label{sect:8-2}


\section{Finite Boolean Algebras}\label{sect:8-3}


\section{Computational Pipeline}\label{sect:8-4}


