% !TEX TS-program = xelatex

\chapter{Symbolic modeling with Julia Plasm}
\label{chapt:5}

Symbolic modeling is a semantic approach to knowledge representation and processing. A  symbolic approach to design to represent information and computation uses names to define the meaning of represented knowledge explicitly. The geometric knowledge is described here by Julia's names, which are chosen suitably for functionals, functions, formal and actual parameters, and finally to objects, fields, classes, attributes, methods, relations, etc. In this chapter, we give many examples of high-level Plasm programming, from topological, linear, and affine operators, to geometric mapping of complexes and grids to generate linearized approximation of curved manifold of intrinsic dimensions 1, 2, and 3. i.e., depending on such number of parameters; say, curves, surfaces, thin, and bulk solids.

\section{ Primitive generators}\label{sect:5-1}


\section{ Plasm topological operators}\label{sect:5-2}


\section{ Linear and affine operators}\label{sect:5-3}


\section{ Manifold mapping}\label{sect:5-4}


\section{ Predefined Plasm functions}\label{sect:5-5}


\section{ Curve, surface, and solid methods}\label{sect:5-6}


