%%%%%%%%%%%%%%%%%%%%%%%%%%%%%%%%%%%
                               
    \documentclass{31x47jw}   
    \usepackage{epsf}         
    \usepackage{latexsym}     
    \usepackage{epsfig}       
    \usepackage{theorem}      
    \usepackage{hyperref}     
    % \psfigdriver{oztex}     
    \newenvironment{titledslide}[1]{\flushleft\begin{slide}\titolo{#1}\vfill}{\vspace{\fill}\end{slide}}
\newcommand{\titolo}[1]{\begin{center}#1\end{center}
 \vspace{-8mm} 
 \rule[10mm]{\textwidth}{0.7mm}}
\newcommand{\epspic}[2]{\epsfxsize=#2mm\begin{center}\leavevmode\epsfbox{#1.ps}\end{center}}
\newcommand{\pl}{{\sf Plasm}} % PLaSM
\newcommand{\simplexn}{{Simple$_{X}^{n}$}} % Simplexn
\newcommand{\fl}{{\sf FL}} % FL

\newenvironment{code}
{\samepage\small\begin{tabbing}aa\=aa\=aa\=aa\=aa\=aa\=aa\=aa\=aa\=aa\=aa\=aa\=aa\=aa\kill}{\end{tabbing}}

\newenvironment{plasm}
{\samepage\tt\begin{quote}
\begin{tabbing}aa\=aa\=aa\=aa\=aa\=aa\=aa\=aa\=aa\=aa\=aa\=aa\=aa\=aa\kill}
{\end{tabbing}\end{quote}}

\newenvironment{smallplasm}
{\begin{quote}\samepage\small\tt
\begin{tabbing}aaa\=aaa\=aaa\=aaa\=aaa\=aaa\=aaa\=aaa\=aaa\=aaa\=aaa\=aaa\=aaa\=aaa\kill}
{\end{tabbing}\end{quote}}

\newenvironment{tinyplasm}
{\scriptsize\tt
\begin{tabbing}aaa\=aaa\=aaa\=aaa\=aaa\=aaa\=aaa\=aaa\=aaa\=aaa\=aaa\=aaa\=aaa\=aaa\kill}
{\end{tabbing}}

\newenvironment{smallerplasm}
{\footnotesize\tt\begin{tabbing}aaa\=aaa\=aaa\=aaa\=aaa\=aaa\=aaa\=aaa\=aaa\=aaa\=aaa\=aaa\=aaa\=aaa\kill}{\end{tabbing}}

\newenvironment{smallestplasm}
{\tt\tiny\begin{tabbing}aaa\=aaa\=aaa\=aaa\=aaa\=aaa\=aaa\=aaa\=aaa\=aaa\=aaa\=aaa\=aaa\=aaa\kill}{\end{tabbing}}

\newenvironment{plasmverb}
{\small\tt\begin{verbatim}}{\end{verbatim}}

% \newenvironment{script}
% {\noindent\rule{5.15in}{0.1mm}\vspace{-0.2cm}\begin{plist}}
% {\end{plist}\vspace{-0.4cm}\rule{5.15in}{0.1mm}\vspace{0.2cm}}

\newenvironment{defscript}
{\begin{minipage}[c]{\linewidth}\begin{plist}}{\end{plist}\end{minipage}}

%\newcommand{\ve}[1]{\bf #1}
\def\E{{\sf I\!E}}

\def\D#1#2{{d^{#1} \over d #2}}
\def\Partial#1#2{{\partial^{#1} \over \partial #2}}
\def\v#1{{\bf #1}}
\def\v#1{{\mbox{\boldmath $#1$}}}
\def\p#1{{\bf #1}}
\def\p#1{{\mbox{\boldmath $#1$}}}
\def\T#1{{\bf #1}}
\def\T#1{{\mbox{\boldmath $#1$}}}
\def\grad#1{\mbox{grad\,}#1}
%\def\vect2 #1 #2{{\left[\begin{array}{cc}#1\\#2\end{array}\right]}}
\def\vet#1{{\left(\begin{array}{cccccccc}#1\end{array}\right)}}
\def\mat#1{{\left(\begin{array}{cccccccc}#1\end{array}\right)}}
\def\deriv#1#2{{d #1 \over d #2}}
\def\pderiv #1 #2{{ \partial #1} \over {\partial #2}}
\def\binomial#1#2{{\left(\!\begin{array}{c}#1\\#2\end{array}\!\right)}}
\def\module#1{{\left|\!\left|#1\right|\!\right|}}


\newtheorem{definition}{Definition}[section]
\newtheorem{Note}{Note}[section]
\newtheorem{rrule}{Rule}[section]
\newtheorem{lemma}{Lemma}[section]
\newtheorem{theorem}{Theorem}[section]
\newtheorem{corollary}{Corollary}[section]

\theorembodyfont{\rmfamily}
\theoremstyle{break}
\newtheorem{example}{Example}[section]
\newtheorem{plist}{Script}[section]

\def\E{\mathbb{I\!\!E}}
\def\R{\mathbb{I\!\!R}}
\def\P{\mathbb{I\!\!P}}
\def\N{\mathbb{I\!\!N}}
\def\Z{\mathbb{I\!\!Z}}
\newcommand{\curl}[0]{\mbox{\rm curl}\,}
\newcommand{\topint}[0]{\mbox{\rm int}\,}
\newcommand{\closure}[0]{\mbox{\rm clos}\,}
\newcommand{\lin}[0]{\mbox{\rm lin}\,}
\newcommand{\aff}[0]{\mbox{\rm aff}\,}
\newcommand{\pos}[0]{\mbox{\rm pos}\,}
\newcommand{\cone}[0]{\mbox{\rm cone}\,}
\newcommand{\conv}[0]{\mbox{\rm conv}\,}
\newcommand{\relint}[0]{\mbox{\rm relint}\,}
\newcommand{\homog}[0]{\mbox{\rm homog}\,}
\newcommand{\ext}[0]{\mbox{\rm ext}\,}
\newcommand{\complement}[0]{\mbox{\rm complement}\,}
\newcommand{\proj}[1]{\mbox{\rm proj}_{#1}\,}
\newcommand{\elim}[1]{\mbox{\rm elim}_{#1}\,}
% \newcommand{\dim}[0]{\mbox{\rm dim}\,}


\def\myDelta{{\scriptscriptstyle{\Delta}}}
\def\one{{\rm \mbox{\small $1$}\!\!1}}

%-----------------------------------------------------------------------------
% \newcommand{\figsdir}[1]{os9:Work:Book:#1:#1fig:}
% \newcommand{\exampledir}[1]{os9:Work:Book:examples:#1:}
\newcommand{\figsdir}[1]{/Users/paoluzzi/BACK/Users/work/Libri/book/#1/}
%-----------------------------------------------------------------------------


%%% Claudio Baldazzi

\newcommand{\bigsymb}[3]{\begin{array}{c}
{\scriptscriptstyle #3}\\[-0.15cm]
{\displaystyle #1} \\[-0.3cm]
{\scriptscriptstyle #2}\end{array}}

\newcommand{\BSP}{\mbox{BSP}}

\newenvironment{lang}
{
\begin{quote}
    \begin{tabbing}
        aaa\=aaa\=aaa\=aaa\=aaa\=aaa\kill
}
{
\end{tabbing}
\end{quote}
}

\newcommand{\note}[1]{\typeout{--------Note: #1}{\sc #1}}
\newcommand{\st}[0]{\;|\;}                           %tale che

% background section
\newcommand{\espace}[0]{E} 
\newcommand{\rspace}[0]{\R}
\newcommand{\vset}[0]{V}
\newcommand{\vsymb}[0]{v}
\newcommand{\fsymb}[0]{f}
\newcommand{\dime}[0]{\mbox{\rm dim}\,}

\newcommand{\cella}[0]{\cell}
\newcommand{\interior}[1]{\stackrel{\circ}{#1}}
\newcommand{\boundary}[0]{\partial}
\newcommand{\interiorcella}[0]{\interior{\cella}}
\newcommand{\insieme}[0]{\rm}
\newcommand{\dcomplesso}[0]{{\cal K}}
\newcommand{\identita}[0]{I} 

\newcommand{\poliedro}[0]{P}  \newcommand{\poliedri}[0]{{\cal P}}
\newcommand{\mappa}[0]{T}     \newcommand{\mappe}[0]{{\cal T}}
\newcommand{\istanza}[0]{I}   \newcommand{\istanze}[0]{{\cal I}}
\newcommand{\complesso}[0]{C} \newcommand{\complessi}[0]{{\cal C}}
\newcommand{\modelli}[0]{{\cal M}}

\newcommand{\Rpoliedro}[0]{\langle P \rangle}
\newcommand{\Ristanza}[0]{\langle I \rangle}
\newcommand{\Rcomplesso}[0]{\langle C \rangle}
\newcommand{\Rmappa}[0]{\langle T \rangle}
\newcommand{\rappresentazioni}[0]{{\cal R}}
\newcommand{\schemaR}[0]{\mbox{\tt r}}
\newcommand{\schemaWR}[0]{\mbox{\tt wr}}

\newcommand{\istanzaB}[0]{\tilde{I}}  \newcommand{\istanzeB}[0]{\tilde{{\cal I}}}
\newcommand{\sequenza}[0]{S}          \newcommand{\sequenze}[0]{{\cal S}}
\newcommand{\Modelli}[0]{{\cal M}^*}
\newcommand{\Rappresentazioni}[0]{{\cal R}^*}
\newcommand{\RistanzaB}[0]{\langle \tilde{I} \rangle}
\newcommand{\Rsequenza}[0]{\langle S \rangle}
\newcommand{\schemaRR}[0]{\mbox{\tt r}^*}
%\newcommand{\diff}[0]{\mbox{\tt V}}
%\newcommand{\Rdiff}[0]{\mbox{\tt V}^*}
\newcommand{\diff}[0]{\mbox{\tt PD}}
\newcommand{\Rdiff}[0]{\mbox{\tt PD}^*}
\newcommand{\oper}[0]{\star}
\newcommand{\operX}[0]{\diamond_\star}
\newcommand{\Rom}[0]{\tt}
\newcommand{\operA}[0]{\operX^{\Rom 1}}
\newcommand{\operB}[0]{\operX^{\Rom 2}}
\newcommand{\operC}[0]{\operX^{\Rom 3}}
\newcommand{\operD}[0]{\operX^{\Rom 4}}

%paper operatori
%scheletro
\newcommand{\kskeleton}[0]{{\tt @k}}
\newcommand{\kskeletonX}[0]{\diamond_{\kskeleton}}
\newcommand{\kskeletonB}[0]{\kskeletonX^{\Rom 2}}
\newcommand{\cell}[0]{c}        % coerenza con  \cella !!!!!!!!!
\newcommand{\rank}[0]{\mbox{\rm rank}\,}

%prodotto
\newcommand{\prodottoX}[0]{\diamond_{\prodotto}}
\newcommand{\sequenzaB}[0]{D}
\newcommand{\poliedroA}[0]{\poliedro}
\newcommand{\poliedroB}[0]{Q}
\newcommand{\poliedroC}[0]{O}
\newcommand{\mappaI}[0]{T_{\identita}}
\newcommand{\matrici}[0]{M}
\newcommand{\prodottoB}[0]{\prodottoX^{\Rom 2}}
\newcommand{\prodottoC}[0]{\prodottoX^{\Rom 3}}
\newcommand{\facce}[0]{F}
\newcommand{\celle}[0]{\dcomplesso}
\newcommand{\cellepoliedroA}[0]{\celle_{\poliedroA}}
\newcommand{\cellepoliedroB}[0]{\celle_{\poliedroB}}
\newcommand{\cellepoliedroC}[0]{\celle_{\poliedroC}}



\newcommand{\umlaut}[1]{\"{#1}}
\newcommand{\Simplex}[0]{\emph{Simple}${}_X^n$\ }
\newcommand{\simplex}[0]{\emph{Simple}${}_X^n$\ }
\newcommand{\rem}[1] {\rule{3mm}{.2mm}\hspace{2mm}{\sc
   #1}\hspace{2mm}\rule{3mm}{.2mm}\hspace{2mm}} 
\newcommand{\prodotto}[2]  {\,\, {}_{#1} \otimes_{#2} \,\,}
\newcommand{\Prodotto}[3]  {\,\, {}_{#1}\!\!\otimes^{#3}_{#2} \,}
\newcommand{\pol}[1]  {{\cal P}^{#1}}
\newcommand{\af}[1]  {{\cal A}^{#1}}
\newcommand{\fa}[1]  {{\cal A}_{#1}}
\newcommand{\ad}[1]{{\cal AD}_{#1}}

\def\vint{\int\!\!\int\!\!\int}
\def\sint{\int\!\!\int}
\def\isum{\sum_{i=1}^n}
\def\asum{\sum_{\alpha=0}^n}
\def\bsum{\sum_{\beta=0}^m}
\def\csum{\sum_{\gamma=0}^p}
\def\sssum{\asum\bsum\csum}
\def\XXa{\sum_{\tau\in T}  |{\bf a} \times {\bf b}| }
\def\XXb{\pmatrix{\alpha\cr  h\cr}}
\def\XXc{\pmatrix{\beta\cr k\cr}}
\def\XXd{\pmatrix{\gamma\cr m\cr}}
\def\XXe{\pmatrix{h\cr i\cr}} 
\def\XXf{\pmatrix{k\cr j\cr}}
\def\XXg{\pmatrix{m\cr l\cr}}
\def\XXh{\pmatrix{p\!\!+\!\!1\cr h\cr}}

\newlength\myheight  %command for \inlinegraphics
\newlength\mydepth
\settototalheight\myheight{Xygp}
\settodepth\mydepth{Xygp}
\setlength\fboxsep{0pt}
\newcommand*\inlinegraphics[1]{%
  \settototalheight\myheight{Xygp}%
  \settodepth\mydepth{Xygp}%
  \raisebox{-\mydepth}{\includegraphics[height=\myheight]{#1}}%
}

%Command for documenting functions

% \newcommand{\function}[5]{
% }}\noindent
% \vspace{0.1cm}\rule{5.15in}{0.1mm}\vspace{-0.2cm}
% \begin{description} \item[{\tt ~#1}] #2\end{description}
% \vspace{-3mm}
% \begin{tabular}[t]{lp{4.2in}}
% \ifthenelse{\equal{#3}{}}{}{{\small Prototype:}&{\small \texttt{#3}}\\}
% \ifthenelse{\equal{#4}{}}{}{{\small Returns:}&{\small \texttt{#4}}}
% \ifthenelse{\equal{#5}{}}{}{\\ {\small Example:}&{\small \texttt{#5}}}
% \end{tabular}}

\newcommand{\function}[6]{
}\noindent
\vspace{0.1cm}\rule{5.15in}{0.1mm}\vspace{-0.4cm}
\begin{description} \item[{\tt ~#1}] {\small #3}\end{description}
\vspace{-2mm}
\begin{tabular}[t]{lp{4.2in}}
\ifthenelse{\equal{#4}{}}{}{{\small \hspace{-0.3mm}Pre/Post conds}&{\small \texttt{#4}}}
\ifthenelse{\equal{#5}{}}{}{{${} \rightarrow {}$}{\small \texttt{#5}}\\}
\ifthenelse{\equal{#6}{}}{}{ {\small \hspace{-1.4mm}Example}&{\small \texttt{#6}}}
\end{tabular}}

%Ignore overfull/underfull warnings

\hbadness=30000
\vbadness=30000
\showboxbreadth=0
\showboxdepth=0
\hfuzz=\maxdimen
\vfuzz=\maxdimen


% macros for book formatting

%\newfloat{bookcopyright}{b}{ext}{}
%\newfloat{floatplasm}{htb}{ext}{}

\newenvironment{script}
{\begin{floatplasm}\noindent\rule{\textwidth}{0.1mm}\vspace{-0.2cm}\begin{plist}}
{\end{plist}\vspace{-0.4cm}\rule{\textwidth}{0.1mm}\vspace{-0.4cm}\end{floatplasm}}

\newenvironment{nofloat}
{\noindent\rule{5.15in}{0.1mm}\vspace{-0.2cm}\begin{plist}}
{\end{plist}\vspace{-0.4cm}\rule{5.15in}{0.1mm}}

%\def\gp4cad{\begin{bookcopyright}
%    \emph{Geometric Programming for Computer-Aided Design}\ \ Alberto Paoluzzi\\
%    \copyright\ 2003 John Wiley \& Sons, Ltd\ \ ISBN 0-471-89942-9
%\end{bookcopyright}}

\makeindex                                         
\def\sysdetails{}
\linewidth = \textwidth

        
    \nonstopmode              
    \begin{document}          
                              
    \setcounter{chapter}{4}   
    \setcounter{page}{13}     
 
%%%%%%%%%%%%%%%%%%%%%%%%%%%%%%%%%%%

\chapter{Elements of differential geometry}
\index{Elements!of differential geometry}
\label{chapt:diffgeom}
\gp4cad

This chapter presents various concepts useful to fully
understand curves and surfaces discussed in later sections.  In
particular, we introduce here the notion of a curve as a point-valued
function of a real variable, the variable-free representation of
functions and the Fr\'echet derivative for functions of one and
several variables.  This derivative gives the framework for
introducing the concepts of manifold, tangent space and vector field
from a geometrical perspective.  Then, more standard topics on
curves are briefly outlined, including tangent, curvature, centre of
curvature and osculating circles.  Finally, some elements of
differential geometry of surfaces are discussed, including first and
second fundamental forms and Gauss curvature.  As usual in this book,
the mathematics is normally given without proofs, but is often coupled
with a \pl\ implementation of the important ideas.  We hope that
the implementation and the examples may help to clarify subtle
mathematical details.


%-------------------------------------------------------------------------
\section{Curves}\index{Curve}
\label{sec:curves}
%-------------------------------------------------------------------------

A \emph{curve} in $ \E^d$ is a point-valued mapping defined by summing
to the origin of a Cartesian system $\{ \p{o}, (\v{e}_{i}) \}$ a
vector-valued function $\v{\alpha}: \R \rightarrow \R^{d}$ of a real
parameter, so that a point of the curve is generated as:
\[
\p{c}(u) = \p{o} + \v{\alpha}(u), \qquad u\in[a,b]\subset\R.
\]
The \emph{image} of the curve is the set $\p{c}[a,b]$ of its $ \E^d$
points.  The \emph{domain} of the curve is the parameter interval
$[a,b]$, i.e.~the set $\{u\in\R | a\leq u\leq b \}$, often norm\-alized to
the standard unit interval~$[0,1]$.

The important part of the curve definition is the vector-valued
function ${\v\alpha}$, so that sometimes we use the word ``curve" for
it.  From a notational viewpoint, we normally use a bold Latin letter,
say $\p{a}$, $\p{b}$ or $\p{c}$, to indicate a map $\R \rightarrow
\E^d$, and a bold Greek letter, say $\p{\alpha}$, $\p{\beta}$ or
$\p{\gamma}$, to indicate a map $\R \rightarrow \R^d$.  This
convention will sometimes be  broken according to the standard usage of
symbols, e.g.~for the \emph{intrinsic triplet} of vector functions
$\R\rightarrow\R^{3}$, that are denoted by $\v{t}$, $\v{n}$ and
$\v{b}$, respectively.

\begin{example}[3D curve]

A curve $\p{c}$ in three-dimensional space has 3 \emph{coordinate
functions}, and is often denoted as
\[
\v{c}(u) = (x(u),  y(u), z(u))^T,
\qquad u\in [a,b]
\]
where ${x}(u) = \v{\alpha}(u) \cdot \v{e}_{1}$, ${y}(u) =
\v{\alpha}(u) \cdot \v{e}_{2}$ and ${z}(u) = \v{\alpha}(u) \cdot
\v{e}_{3}$.  The \emph{tangent} vector function $\p{t}$ is defined
by
\[
\p{t}(u) = \deriv{}{u} \v{c}(u) = \deriv{}{u} \v{\alpha}(u) =
(x'(u), y'(u), z'(u))^T, \qquad u\in [a,b].
\]

\end{example}


% \begin{example}[Circle]
% 
% The following
% expression generates a polygonal approximation with 36 line segments
% of the unit circle centered in the origin, i.e.~of the set 
% $\v{c}[0,2\pi]$, with \v{c}(u) = (0, 0) + (\cos u, \sin u).
% 
%   
% \begin{smallplasm} 
% MAP:([COS, SIN] $\sim$ S1):(interval:<0,2*PI>:36)
% \end{smallplasm} 
% 
% \end{example}



\subsection{Variable-free notation}\index{Variable!-free notation}
\label{sec:5:free}

Accordingly with the functional approach of the \pl\ language, we
often denote a 3D curve, as well as its derivative curves, by using
the variable-free notation:
\[
\v{c} = (x,  y, z)^T
\]
with ${x} = \v{\alpha} \cdot \v{e}_{1}$, ${y} = \v{\alpha} \cdot
\v{e}_{2}$, and ${z} = \v{\alpha} \cdot \v{e}_{3}$.%, and so on.

It should be clearly understood that $x, y, z$ %as  $x^{u}, y^{u}, z^{u}$
are here maps $\R\rightarrow\R$ and that each $\v{e}_{i}$ has the
constant maps $\underline{0}:\R\rightarrow 0$ and
$\underline{1}:\R\rightarrow 1$ as components.

Analogously, we write a curve $\p{c}:\R\rightarrow \E^{d}$, as
$\p{c}=\p{o}+\v{\alpha}$, with $\v{\alpha}=(\alpha_{i})$, where
$\alpha_{i}: \R\rightarrow\R$, for all $i$.  The variable-free
notation, where functions are directly added and multiplied, exactly
like numbers, is very useful for quickly and easily implementing
curves and surfaces in our language.

\paragraph{Useful maps}\index{Useful maps}

Some special maps are needed to perform such variable-free calculus
with functions.  As the reader already knows, they have a direct
translation in \pl.
\begin{enumerate}
    
    \item    
    $\textrm{id}: \R\rightarrow\R; x\mapsto x$
    \hfill ({identity})
    
    \item    
    $\underline{c}: \R\rightarrow\R; x\mapsto c$
    \hfill (constant)
    
    \item    
    $\sigma: \{1,\ldots,d\}\times \R^{d}\rightarrow\R;
    (i,(x_{1},\ldots,x_{d}))\mapsto x_{i}$ \hfill (selection)

\end{enumerate}

A computer scientist would probably prefer the following 
specification, just to point out that $\sigma$ is often used as a 
\emph{partial} function, i.e.~a function which may be applied to a 
subset of its arguments:
\begin{enumerate}
    
    \item[3.]
    $\sigma: \{1,\ldots,d\}\rightarrow(\R^{d}\rightarrow\R);
    i\mapsto((x_{1},\ldots,x_{d})\mapsto x_{i})$ \hfill (selection)

\end{enumerate}

Actually, the \fl\ primitives \texttt{ID}, \texttt{K} and \texttt{SEL}
used by the \pl\ language have no domain restrictions, and can be
applied to any type of data objects.

\paragraph{Algebraic operations}\index{Algebraic!operations}

We also need to recall how to perform algebraic operations in the
linear algebra of maps $\R\rightarrow\R$.  For each map
$\v{\alpha},\v{\beta}: \R\rightarrow\R$ and each scalar $a\in\R$
\[
\v{\alpha} + \v{\beta}: u \mapsto \v{\alpha}(u) + \v{\beta}(u),
\quad 
\v{\alpha}  \v{\beta} : u \mapsto \v{\alpha}(u)  \v{\beta}(u), 
\quad
a\,\v{\beta} : u \mapsto a\,\v{\beta}(u).
\]

Consequently we have that
\[
\v{\alpha} - \v{\beta} : u \mapsto \v{\alpha}(u) - \v{\beta}(u),
\quad 
\v{\alpha} / \v{\beta} : u \mapsto \v{\alpha}(u) / \v{\beta} (u).
\]

\paragraph{Coordinate representation}\index{Coordinate!representation}

Finally, remember that the coordinate functions of a curve $\v{\alpha}
= (\alpha_{i})$ are maps $\R\rightarrow\R$.  The variable-free
vector notation stands for the linear combination of coordinate
functions with the basis vectors of the target space:
\[
({\alpha}_{1}, \cdots, {\alpha}_{d})^{T}:\
\R\rightarrow\R^{d} ;\ u \mapsto
\sum_{i=1}^{d} {\alpha}_{i} \v{e}_{i}.
\]


\begin{example}[Circular arc]
\label{ex:circumference}
Some different curves are given here.  They have the same image in $
\E^{2}$ but different coordinate representation in the space of
functions $\R\rightarrow\R$.  All such curves generate a circular arc
of unit radius centered at the origin.

% \begin{itemize}
%     \item[]
\begin{enumerate}
    
\item  trigonometric representation:

\[
\p{\alpha}(u) = \left(\cos \left({\pi \over 2}u\right), \sin \left({\pi \over
2}u\right)\right)^T \qquad u\in[0,1]
\]

\item  rational representation:
\[
\p{\beta}(u) = \left({1-u^2 \over 1+u^2}, {2u \over 1+u^2}\right)^T
\qquad u\in[0,1]
\]


\item  Cartesian representation:

\[
\p{\gamma}(u) = \left(u, \sqrt{1-u^2}\right)^T
\qquad u\in[0,1]
\]

\end{enumerate}
% \end{itemize}

It is possible to verify that the image sets of such curves coincide, 
i.e.~that $\p{\alpha}[0,1]=\p{\beta}[0,1]=\p{\gamma}[0,1]$.

\end{example}


\begin{example}[Variable-free circular arc]
\label{ex:freearc}
    
It may be useful to give the variable-free representation of the three
maps on the $[0,1]$ interval shown in Example~\ref{ex:circumference},
that is exactly the representation we need to give a \pl\
implementation of such maps, provided in Script~\ref{script:freearc}:
\begin{eqnarray}
\p{\alpha} &=& \left(\cos \circ \left( \underline{\pi \over 2} 
\textrm{id}\right), \
\sin \circ \left( \underline{\pi \over 2} \textrm{id} \right) \right)^T
\\
\p{\beta} &=& \left(\frac{\underline{1} - 
\textrm{id}^2 }{ \underline{1}+ \textrm{id}^2}, \
{\underline{2} \,\textrm{id} \over \underline{1}+ \textrm{id}^2} 
\right)^T
\\
\p{\gamma} &=& \left( \textrm{id}, \ \textrm{id}^{1\over 2}\circ 
(\underline{1}- \textrm{id}^2) \right)^T
\label{eq:5:alpha}
\end{eqnarray}

\end{example}


\paragraph{Toolbox}\index{Toolbox}

Some predicates and functions needed by the operators in this
chapter are given in Script~\ref{script:5:toolbox}.  In particular,
the \texttt{interval} operator provides a simplicial decomposition with
\texttt{n} elements of the real interval $[\texttt{a},\texttt{b}]$,
whereas the \texttt{interval2D} operator returns a decomposition with
$\texttt{n1} \times \texttt{n2}$ subintervals of the domain
$[\texttt{a1},\texttt{b1}]\times
[\texttt{a2},\texttt{b2}]\subset\R^{2}$.

\begin{script}[Toolbox] 
\begin{smallplasm} 
% DEF IsPoint \index{Library functions!Vectors library!{\tt  Ispoint}} = IsSeqOf:IsReal;\\
% DEF IsVect \index{Library functions!Vectors library!{\tt  Isvect}} ~= IsSeqOf:IsReal;\\
% DEF SQR \index{Library functions!General library!{\tt  Sqr}} ~~~~= ID * ID;\\
DEF interval (a,b::IsReal)(n::IsIntPos) = 
  (T:1:a  $\sim$  QUOTE  $\sim$  \#:n):((b-a)/n);\\
DEF interval2D (a1,a2,b1,b2::IsReal)(n1,n2::IsIntPos) = \+\\
  interval:<a1,b1>:n1 * interval:<a2,b2>:n2;
\end{smallplasm} 
\label{script:5:toolbox}
\end{script}
Few other functions of general utility are used in the remainder of
this chapter.  In particular, the \texttt{SQR} function, that returns
its squared input, was defined in Script~\ref{script:UnitVect};
the predicates \texttt{IsVect} and \texttt{IsPoint} were given in
Script~\ref{script:IsVect} and~\ref{script:3:parametricplane}, 
respectively.  The vector operations used here were
defined in Chapter~3.

\paragraph{Implementation}\index{Implementation!circular arc maps}

The circle segment representations of Example~\ref{ex:freearc} are directly
used in the \pl\ implementation of curves in
Script~\ref{script:freearc}.  To understand the implementation, notice
that we generate a polyhedral complex by mapping the curve vector
function (either $\v{\alpha}$, $\v{\beta}$ or $\v{\gamma}$ of
Example~\ref{ex:freearc}) on the polyhedral representation of the
$[0,1]\subset\R$ domain.

According to the semantics of the \texttt{MAP} function, the curve
mapping is applied to all vertices of a simplicial decomposition of
the polyhedral domain.  But all vertices are represented as
\emph{sequences} of coordinates, say \texttt{<$u$>} for a curve, so
that in order to act on $u$ the mapping must necessarily \emph{select}
it from the sequence.

Hence we might substitute each $id$ function instance with the \pl\
denotation \texttt{S1} for the $\sigma(1)$ selector function.  

Exactly the same result is obtained by using either
$\v{\alpha}\circ\sigma(1), \v{\beta}\circ\sigma(1)$ or
$\v{\gamma}\circ\sigma(1)$, as done in the following code. 

\begin{script} [Circular arc maps]
\begin{smallplasm}
DEF SQRT~~\== ID ** K:(1/2);\\[0.3cm]
DEF alpha \>= < cos $\sim$ (K:(PI/2) * ID), sin $\sim$ (K:(PI/2) * 
ID) >; \\
DEF beta  \>= < (K:1 - SQR)/(K:1 + SQR),  (K:2 * ID)/(K:1 + SQR) >;\\
DEF gamma \>= < ID, SQRT $\sim$ (K:1 - SQR) >;\\[0.3cm]

MAP:(CONS:alpha $\sim$ S1):(interval:<0,1>:10);\\
MAP:(CONS:beta~ $\sim$ S1):(interval:<0,1>:10);\\
MAP:(CONS:gamma $\sim$ S1):(interval:<0,1>:10);
\end{smallplasm}
\label{script:freearc}
\end{script}

\paragraph{Remarks}\index{Coordinate!functions}

Let us note that, e.g., \texttt{alpha} is a \emph{sequence} of
coordinate functions.  Conversely, \texttt{CONS:alpha} is the correct
implementation of the \emph{vector-valued} function $\v{\alpha}$,
which only can be \emph{composed} with other functions, say
\texttt{S1}.

Notice also that \texttt{SQR} (square), given in
Script~\ref{script:UnitVect}, is the \pl\ implementation of the
$\mbox{id}^{2}$ function and that the language explicitly requires the
operator \texttt{*} to denote the product of functions.

Finally, we remark that \texttt{SQRT} (square root), which is actually
primitive in \pl, can be also defined easily using standard algebraic
rules, where \texttt{**} is the predefined power operator.

    

\begin{example}[Comparing parametrizations]
\label{ex:5:compare}

The parametrizations of the arcs generated by maps $\v{\alpha}$,
$\v{\beta}$ and $\v{\gamma}$ are quite different.  It is possible to
see this fact by looking at Figure~\ref{fig:arcs}, where the graphical
markers associated with the mapped image of each point of a uniform
sampling of the curve domain generated by
Script~\ref{script:freearcpar} are shown.  The \texttt{polymarker} and
\texttt{markerSize} implementation is given in
Script~\ref{script:polymarker}.  Notice that the functional expression
\begin{smallplasm}
    polymarker:1 $\sim$ S1 $\sim$ UKPOL
\end{smallplasm}
converts its polyhedral input into a triplet of vertices, cells and
polyhedra, then extracts the vertices, and finally gives them as
arguments to the function \texttt{polymarker:1}, which attaches a
marker of type \texttt{1} to each point.

\begin{script} [Curve sampling marking]
\begin{smallplasm}
DEF markerSize \index{Library functions!Primitives library!{\tt  Markersize}} = 0.05;\\    
DEF markers = polymarker:1 $\sim$ S1 $\sim$ UKPOL; \\[0.3cm]
(markers $\sim$ MAP:(CONS:alpha $\sim$ S1)):(interval:<0,1>:10);\\
(markers $\sim$ MAP:(CONS:beta~ $\sim$ S1)):(interval:<0,1>:10);\\
(markers $\sim$ MAP:(CONS:gamma $\sim$ S1)):(interval:<0,1>:10);
\end{smallplasm}
\label{script:freearcpar}
\end{script}

\begin{figure}[htb]
    \centering
    \begin{minipage}[b]{0.32\linewidth}
            \centering\epsfig{file=\figsdir{cap05}arc1.eps,width=0.8\linewidth}
    \end{minipage}
    \hfill
    \begin{minipage}[b]{0.32\linewidth}
            \centering\epsfig{file=\figsdir{cap05}arc2.eps,width=0.8\linewidth}
    \end{minipage}
    \hfill
    \begin{minipage}[b]{0.32\linewidth}
            \centering\epsfig{file=\figsdir{cap05}arc3.eps,width=0.8\linewidth}
    \end{minipage}
\caption{Trigonometric, rational and Cartesian parametrization 
\label{fig:arcs}}
\end{figure}

\end{example}


\begin{example}[Coordinate maps]
\label{ex:5:coordinatemaps}
    
It should be remembered that curves, as~$\v{\alpha}$, $\v{\beta}$ and
$\v{\gamma}$ in the previous example, are vector-valued functions.  In order
to obtain their coordinate maps, say $\alpha_{i}:\R\rightarrow\R$, a
composition with the appropriate selector function is needed:
\[
\alpha_{i} = \sigma(i)\circ\v{\alpha}
\]

The conversion from a 3D vector-valued function \texttt{ curve $:=$
CONS:alpha } to the sequence of its coordinate functions may be obtained
in \pl\ as:
\begin{smallplasm}
< S1 $\sim$ curve, S2 $\sim$ curve, S3 $\sim$ curve >;
\end{smallplasm}
Such an approach is quite expensive and inefficient, because the curve
function is repeatedly evaluated on data points to get its three
component functions.  So, for the sake of efficiency, we suggest 
maintaining a coordinate representation as a \emph{sequence} of
scalar-valued functions, and then \texttt{CONS} it into a single
\emph{vector-valued} function only when it is strictly necessary.

Anyway, an operator \texttt{Curve2MapVect} is given in
Script~\ref{script:Curve2MapVect}.  This operator, when applied to a
vector-valued map $\R\rightarrow\E^{d}$ with arbitrary $d$, will
return the sequence of its $d$ coordinate functions.  

\begin{script} [Curve to vector of maps]
\begin{smallplasm}
DEF Curve2MapVect \index{Library functions!Curves library!{\tt  Curve2mapvect}} (curve::IsFun) = AA:COMP:(selectors DISTR curve)\\
WHERE\+\\
  selectors = AA:SEL:(1..d),\\
  d = LEN:(curve:<0>)\-\\
END;\\[0.3cm]

Curve2MapVect:curve $\equiv$ < curve$_{1}$ , curve$_{2}$ >
\end{smallplasm}
\label{script:Curve2MapVect}
\end{script}

\end{example}




\subsection{Reparametrization}\index{Reparametrization}
\label{sec:8:Reparametrization}

A \emph{smooth curve} is defined as a curve whose coordinate functions
are smooth (see Section~\ref{sec:5:differentiation}) i.e.~can be
derived as many times it is needed.  If $\v{c}:I\rightarrow \E^{d}$,
with $I\subset\R$, is a smooth curve and $\rho:I\rightarrow I$ is a
smooth invertible function, i.e.~a \emph{diffeomorphism}, then also
\[
\v{c}_{\rho} = \v{c} \circ \rho
\]
is a smooth curve. It is called a \emph{reparametrization} of $\v{c}$.

A very simple reparametrization is the \emph{change of origin}.  For
example $\v{c}_{\rho}$ is called a {change of origin} when
\[
\rho=\textrm{id}+\underline{c}.
\]
A reparametrization $\v{c}_{\tau}$ by an affine
function 
\[
\tau=\underline{a}\,\textrm{id}+\underline{c},
\]
with $a\not= 0$, is called an \emph{affine reparametrization}.

\begin{example}[Circle reparametrization]
    
The circle with the center in the origin and unit 
radius may be parametrized on different intervals:
\begin{eqnarray*}
\p{c}_{1}(u) &=& \vet{\cos u & \sin u}, 
\qquad\qquad\qquad\qquad~~~ u\in [0,2\pi]
\\
\p{c}_{2}(u) &=& \vet{\cos (2\pi u) & 
\sin (2\pi u)}, 
\qquad\qquad\qquad u\in [0,1]
\label{ex:reparametrization}
\end{eqnarray*}
or with a different starting point:
\begin{eqnarray*}
\p{c}_{3}(u) &=& \vet{\cos (2\pi u+\frac{\pi}{2}) & 
\sin (2\pi u+\frac{\pi}{2})}, 
\qquad u\in [0,1]
\label{ex:reparametrization2}
\end{eqnarray*}
The reparametrization becomes evident if we use a variable-free 
representation:
\begin{eqnarray*}
\p{c}_{1} &=& \vet{\cos & \sin } , 
\\
\p{c}_{2} &=& \vet{\cos & \sin } \circ 
(\underline{2}\,\underline{\pi}\, \textrm{id} ), 
\\
\p{c}_{3} &=& \vet{\cos & \sin } \circ 
(\underline{2}\,\underline{\pi}\, \textrm{id} + 
{\underline{\pi}}/{\underline{2}
}).
\end{eqnarray*}

\end{example}


\begin{example}[Reparametrization of a parabola]
\label{ex:parabola}
Several curves may have the same image.  Two different
parametrizations of the same subset of parabola $y = x^2$ are given
below:
\begin{eqnarray*}
\p{p}_1(u) &=&  \vet{u & u^2}, \qquad u\in[0,2], \\
\p{p}_2(u) &=&  \vet{2u & 4u^2}, \qquad u\in[0,1], \\
% \p{p}_3(u) &=&  \vet{u^{1/3} & u^{2/3}}, \qquad u\in[0,8].  
\end{eqnarray*}
Using the variable-free notation, we have:
% the reparametrizations:
\begin{eqnarray*}
\p{p}_1 &=&  \vet{\textrm{id} & \textrm{id}^2}, \\
\p{p}_2 &=&  \vet{\textrm{id} & \textrm{id}^2}
\circ \underline{2} \,\textrm{id}. %\\ %\qquad
% \p{p}_3[0,8] &=&  \p{p}_1
% \circ \frac{\underline{1} }{\textrm{id}^{3}} . %\qquad
\end{eqnarray*}

A very similar \pl\ implementation of curve $\p{p}_2$ follows.  The
\texttt{interval} function is defined in Script~\ref{script:5:toolbox}.

\begin{script} [Parabola]
\begin{smallplasm}
DEF p2 = [ID, ID * ID] $\sim$ (K:2 * S1);\\[0.3cm]
MAP:p2:(interval:<0,1>:30);
\end{smallplasm}
\label{script:parabola}
\end{script}

\end{example}

\subsection{Orientation}\index{Orientation}
\label{subsec:5:orientation}

Two curves with the same image can be distinguished by the sense in 
which the image is traversed for increasing values of the parameter. Two 
curves which are traversed in the same way are said to have the same 
\emph{orientation}. Actually, an orientation is an equivalence class 
of parametrizations.

A \emph{reversed orientation} of a curve $\p{c}:\R\rightarrow
\E^{d}$, with image $\p{c}[a,b]$, is given by the affine
reparametrization $\p{c}_{\lambda}=\p{c}\circ\lambda$, where
${\lambda}:\R\rightarrow\R$ such that $x\mapsto-x+(a+b)$.  This map
can be written as:
\[
\lambda = \underline{-1}\, \textrm{id} + (\underline{a+b}).
\]


\begin{example}[Reversing orientation]
\label{ex:rev}
    
It is useful to have a \pl\ function \texttt{REV}, which reverses
the orientation of any curve parametrized on the interval
$[a,b]\subset\R$.  The \texttt{REV} function will therefore be 
abstracted with respect to the bounds \texttt{a} and
\texttt{b}, which are real numbers.

In Script~\ref{script:5:rev} we also give a polyhedral approximation
of the boundary of unit circle centered in the
origin, as seen from the angle interval $[0,\frac{\pi}{2}]$.  The curve
is a map from $[0,1]$ with reversed orientation.  

\begin{script} 
\begin{smallplasm}
DEF REV \index{Library functions!Curves library!{\tt  Rev}} (a,b::IsReal) = K:-1 * ID + K:(a+b);\\%[0.3cm]

DEF alpha = [ COS, SIN ] $\sim$ (K:(PI/2) * ID);\\[0.3cm]

MAP:(alpha $\sim$ REV:<0,1> $\sim$ S1):(interval:<0,1>:20);
\end{smallplasm}
\label{script:5:rev}
\end{script}

\end{example}

\section{Differentiation}\index{Differentiation}
\label{sec:5:differentiation}


Let $\v{\phi}: U \rightarrow F$ be a \emph{continuous} map, with $E,
F$ normed linear spaces, $U\subset E$ open set, and $\p{x}\in U$.
    
\begin{definition}
    The map $\v{\phi}$ is \emph{differentiable} at $\p{x}$ if there is a
    \emph{linear} map $\T{L}_{x}: E \rightarrow F$ which approximates
    $\v{\phi}$ at $\p{x}$, i.e. such that
    
    \[
    (\v{\phi}(\p{x}+\v{h}) - \v{\phi}(\p{x})) - \T{L}_{\p{x}}(\v{h}) =
    o(\module{\v{h}})
    \]
\end{definition}
where $o(\module{\v{h}})$ means that the left-hand side approaches zero
faster than $\module{\v{h}}$.  In general, $\T{L}_{\p{x}}$ depends on
$\p{x}$.  When it exists, it is unique.  If $\v{\phi}$ is differentiable at
every $\v{x}\in U$ we say that $\v{\phi}$ is differentiable in $U$.

$\T{L}_{x}$ is a linear map in $\lin(E;F)$.  It is called the
\emph{(Fr\'echet-)derivative of $\v{\phi}$ at $\p{x}$} and is denoted as 
$D\v{\phi}(\p{x})$. 
Hence we can write:

    \[
    \v{\phi}(\p{x}+\v{h}) - \v{\phi}(\p{x}) = D\v{\phi}(\p{x})(\v{h}) + o(\v{h})
    \]

The map $D\v{\phi}: U \rightarrow \lin(E;F)$ is called the
\emph{derivative of $\v{\phi}$}.

A map $\v{\phi}: U\rightarrow F$ is \emph{continuously differentiable} if it
is differentiable and $D\v{\phi} \in C^{0}(U;\lin(E,F))$.  We also say that
$\v{\phi}$ is of class $C^{1}$.  Analogously, 
\[
C^{n} = \{ \v{\phi}\in C^{n-1}: D\v{\phi} \in C^{n-1} \}.
\]
is the class of functions which are $n$-times differentiable.  A
\emph{smooth} function is a function of class $C^{\infty}$.
    
\subsection{Real-valued maps of a real variable}\index{Real-valued maps!of a real variable}

For real maps of a single real variable we have $E = F = \R$.  In this
case the elementary notion of derivative can be recovered from the
above definition.

The function $\phi : \R \rightarrow \R$ is said \emph{differentiable} at $x$
if the limit
\[
\lim_{h \rightarrow 0}\ {1\over h}\ (\phi(x+h) - \phi(x))
\]
exists and is finite.  
That limit is called the \emph{derivative} of $\phi$ at $x$, and is
denoted by $\phi'(x)$.  Its value $\lambda$ is both a number in $\R$ and
a linear map in $\lin(\R,\R)$ via the canonical isomorphism
$\lambda\mapsto\lambda(1)=\lambda$, such that $y\mapsto \lambda y$.

Hence we may write for the ordinary derivative $\phi'(x) = D\phi(x)(1)$.  This
value is the \emph{slope} of the tangent to the graph of $\phi$ at point
$(x,\phi(x))$.


Because of the canonical isomorphism and the consequent identification
of $\R$ with the dual space $\R^{*}$  of linear maps
$\R\rightarrow\R$, we may also write, for $\phi:\R\rightarrow\R$
\[
D\phi:\R\rightarrow\R; x \mapsto \phi'(x).
\]


\paragraph{Implementation}
\index{Implementation!derivative maps!real-valued, of a real variable}

The value of $\phi'(x)$ can be computed, according to its definition, by
the function \texttt{deriv0} of Script~\ref{script:5:D11}, with
\texttt{h}  a suitably ``small" number.  A \emph{much} better numerical
approximation to the slope of tangent to graph of $\phi$ at $x$ is given
by the central difference implemented by function \texttt{deriv1}, as
shown by the following approximations of $\sin' {\pi\over 3}=\cos
{\pi\over 3}={1\over 2}$:
\begin{smallplasm}
deriv0:1E-4:SIN:(PI/3)  $\equiv$ 0.499956697895\\
deriv1:1E-4:SIN:(PI/3)  $\equiv$ 0.499999999166
\end{smallplasm}

\begin{script}[Derivation of $\phi:\R\rightarrow\R$ at ${x}$]
\begin{smallplasm}
DEF deriv0(h::IsReal)(f::IsFun)(x::IsReal) = %\+\\
    (f:(x + h) - f:x)/h;\\%[0.3cm]
    
DEF deriv1(h::IsReal)(f::IsFun)(x::IsReal) = %\+\\
    (f:(x + h) - f:(x - h))/(2*h);\\[0.3cm]
    
DEF D11 (f::IsFun)(x::IsReal) = deriv1:1E-4:f:x;
\end{smallplasm}
\label{script:5:D11}
\end{script}

The \texttt{D11} operator is given to compute the derivative of 
functions $\R\rightarrow\R$ at a point $x$.  Such specialized
operator will be needed in Script~\ref{script:5:D} to implement a
\emph{generalized} derivation operator \texttt{D}, to be used to
derive maps between spaces of any dimensions.  Hence in the
rest of the chapter we only make use of this generalized operator.

The differentiation operator \texttt{D} must be applied first to a
function, and then to a real number, in order to return a real number. 
When it is applied only to a function, it returns the \emph{derivative
function} of its argument:

\begin{smallplasm}
D:COS:0 $\equiv$ 0.0 \\
D:COS:(PI/2) $\equiv$ -0.9999999983332231\\
D:COS $\equiv$ An-Anonymous-Function
\end{smallplasm}



\begin{example}[Map of the derivate functions]
    
In Figure~\ref{figure:5:ex1} it is shown the graph of the $\cos$
function, together with the graph of the $D\cos$ function.  For this
purpose we apply, in Script~\ref{script:5:interval}, the \texttt{MAP}
primitive to function \texttt{[ID, COS] $\circ$ $\sigma(1)$} and to
function \texttt{[ID, D:COS] $\circ$ $\sigma(1)$}, respectively, and
then to a 1D polyhedral complex which defines a partition of the
domain $[-\pi,\pi]$ with 60 subintervals.

Notice that the same approach is used to generate the graphs of $x$
and $y$ axes, by mapping the \texttt{[ID, K:0] $\circ$ $\sigma(1)$}
and \texttt{[K:0, ID] $\circ$ $\sigma(1)$} functions, respectively. 
The generator \texttt{interval} of the domain partition is given in
Script~\ref{script:5:toolbox}.
    
\begin{script}
\begin{smallplasm}
STRUCT:< \+\\
  MAP:([ID, COS] $\sim$ S1):(interval:<-:PI,PI>:60),\\
  MAP:([ID, D:COS] $\sim$ S1):(interval:<-:PI,PI>:60),\\
  MAP:([ID, K:0] $\sim$ S1):(interval:<-:PI,PI>:1),\\
  MAP:([K:0, ID] $\sim$ S1):(interval:<-1,1>:1) >;
\end{smallplasm}
\label{script:5:interval}
\end{script}


\begin{figure}[htb]
\centering\epsfig{file=\figsdir{cap05}ex1.eps,width=0.6\linewidth}
\caption{Graphs of $\cos$ and $D\cos$ functions
\label{figure:5:ex1}}
\end{figure}

    
\end{example}



\subsubsection{Higher-order derivatives}\index{Higher-order!derivatives}

The differentiation operator $D$ can be applied repeatedly on a
{smooth} function $\phi:\R\rightarrow\R$.  In particular:
\[
D^{2}\, \phi := 
(D \circ D)\, \phi = D\, 
{\phi}' = (\phi')' =: \phi^{(2)}.
\]
More in general, for each integer $n>1$,  
\[
\phi^{(n)} := D^{n}\, \phi = D\, \phi^{(n-1)}.
\]

\begin{example}[Polynomials]

The set of polynomial functions of degree $\leq n$ with coefficients
in $\R$ is often denoted as $P_{n}[\R]$.  Functions $\mu\in
P_{n}[\R]$, where
\[
\mu: \R\rightarrow\R ;\ x\mapsto\sum_{i=0}^{n}a_{i}x^{i}, \qquad 
a_{i}\in\R,
\]
are an important example of \emph{smooth functions}, and constitute a
vector space of dimension $n+1$.  In later chapters we will see that
some special bases of such vector space play a very important role in
defining \emph{free-form} curves and surfaces used by CAD systems.

It is very easy to see that the derivative of a polynomial of degree
$n$ is a polynomial of degree $n-1$.  

In other words, if $\mu\in P_{n}[\R]$, then $D\mu\in P_{n-1}[\R]$. 
This fact implies that
\[
P_{n}[\R] \supset D(P_{n}[\R]) \supset \cdots \supset 
D^{n-1}(P_{n}[\R]) \supset D^{n}(P_{n}[\R]).
\]


\end{example}

\begin{example}[Graph of derivatives of $\cos$]
    
The graphs of functions $\cos$, $D\cos$, $D^{2}\cos$ and $D^{3}\cos$
on a partition of the domain $[-\pi, \pi]$ with $60$ subintervals are
produced by the last expression in Script~\ref{script:5:dom}, and are
shown in Figure~\ref{figure:5:ex6}.
    
\begin{script}
\begin{smallplasm}
DEF dom = interval:<-:PI,PI>:60;\\[0.3cm]

STRUCT:<\+\\
  MAP:([ID, COS] $\sim$ S1):dom,\\
  MAP:([ID, D:COS] $\sim$ S1):dom,\\
  MAP:([ID, (D $\sim$ D):COS] $\sim$ S1):dom,\\
  MAP:([ID, (D $\sim$ D $\sim$ D):COS] $\sim$ S1):dom >;
\end{smallplasm}
\label{script:5:dom}
\end{script}


\begin{figure}[htb]
\centering\epsfig{file=\figsdir{cap05}ex6.eps,width=0.6\linewidth}
\caption{Graphs of $\cos$ function and of its first three derivatives
\label{figure:5:ex6}}
\end{figure}

\end{example}




\subsubsection{Useful properties of derivatives} 

Let us consider $E = F = \R$.  The following properties hold for maps in
$C^{1}(\R;\R)$, in the following denoted as $C^{1}$.  Actually, such
properties hold also in more general settings.

\paragraph{Linearity}\index{Linearity}

For all $a,b\in\R$ and all $\phi,\psi\in C^{1}$, the \emph{linear 
combination} map $a
\phi + b \psi \in C^{1}$, with
\[
D(a \phi + b \psi) = a D\phi + b D\psi.
\]

\paragraph{Chain rule}\index{Chain rule}

For all $\phi,\psi\in C^{1}$, the \emph{compound} map $\phi
\circ \psi \in C^{1}$, with
\[
D(\phi \circ \psi) = (D\phi \circ \psi) D\psi.
\]


\paragraph{Leibnitz's rule}\index{Leibnitz's rule}

For all $\phi,\psi\in C^{1}$, the \emph{product} map $\phi \psi
\in C^{1}$, with
\[
D(\phi \psi) = (D\phi) \psi + \phi (D\psi).
\]

\begin{example}
    
In Figure~\ref{figure:5:ex61} we show the graph of $\sin\circ\cos$
function and of its three first derivatives $D(\sin\circ\cos),
D^{2}(\sin\circ\cos)$ and $D^{3}(\sin\circ\cos)$, as generated by 
Script~\ref{script:5:lin}.
    
\begin{script}[A few derivatives of $\sin\circ\cos$]
\begin{smallplasm}
STRUCT:< \+\\
  MAP:([ID, (SIN $\sim$ COS)] $\sim$ S1):dom,\\
  MAP:([ID, D:(SIN $\sim$ COS)] $\sim$ S1):dom,\\
  MAP:([ID, (D $\sim$ D):(SIN $\sim$ COS)] $\sim$ S1):dom,\\
  MAP:([ID, (D $\sim$ D $\sim$ D):(SIN $\sim$ COS)] $\sim$ S1):dom\-\\
>;
\end{smallplasm}
\label{script:5:lin}
\end{script}
The reader may check that the same graph generated by
Script~\ref{script:5:lin} and shown in Figure~\ref{figure:5:ex61} can
be produced by substituting the right-hand side of the following
expressions to the left-hand side.  The first equivalence is produced
by the chain rule; the second equivalence is generated by using both
chain and Leibnitz's rules.
\begin{smallplasm}
D:(SIN $\sim$ COS) $\equiv$ (D:SIN  $\sim$  COS) * D:COS\\[0.3cm]
(D $\sim$ D):(SIN $\sim$ COS) $\equiv$ \+\\\
  D:(D:SIN $\sim$ COS) * D:COS  + (D:SIN $\sim$ COS) * D:(D:COS)
\end{smallplasm}
An equivalent symbolic expression for \texttt{(D $\sim$ D $\sim$
D):(SIN $\sim$ COS)} would produce several lines of code, as the
diligent reader might check.  Furthermore, it would require a longer
computation and would produce a greater numeric error when evaluated
on actual data.

\begin{figure}[htb]
\centering\epsfig{file=\figsdir{cap05}ex61.eps,width=0.55\linewidth}
\caption{Graphs of $\phi=\sin\circ\cos$ and of its first three derivatives 
on $[-\pi, \pi]$
\label{figure:5:ex61}}
\end{figure}

    
\end{example}

\subsection{Vector-valued maps of a real variable}\index{Vector!-valued maps of a real variable}

In this case we have $E = \R$, $F = \R^{m}$, and
$\v{\phi}=(\phi_{1},\ldots,\phi_{m})$ is a vector-valued
\emph{continuous} map $\R \rightarrow \R^{m}$.  In other words,
$\v{\phi} \in C^{0}(\R;\R^{m})$.  Such a map is called a
\emph{curve} in $\R^{m}$.

At each $u\in\R$ the derivative $D\v{\phi}(u)$, if it exists, is a
\emph{linear} map $\R \rightarrow \R^{m}$.

By linearity it is, for all $s\in\R$:
\[
D\v{\phi}(u)(s) = D\v{\phi}(u) (s 1) = s D\v{\phi}(u)(1) = s \v{\phi}'(u) = s
(\phi'_{1}(u), \ldots, \phi'_{m}(u))
\]

\paragraph{Implementation}\index{Implementation!derivative maps!vector-valued, of a real variable}

The function \texttt{D1m}, given in Script~\ref{script:5:D1m}, which
implements this derivative, is very easy to write down by considering
that $D\v{\phi}(u)(s)$ is just a scalar multiple of the vector
$(\phi'_{1}(u), \ldots, \phi'_{m}(u))$ whose components are the
derivatives at $u$ of the scalar functions $\phi_{1},\ldots,
\phi_{m}\in C^{1}(\R;\R)$.  

Notice that the \texttt{scalarVectProd} operator, defined in
Script~\ref{script:2:scalarVectProd} and normally used to multiply a
scalar number times an ordered $d$-tuple of scalars, is here used to
multiply a scalar function times an ordered $d$-tuple of functions. 
Such useful behavior of vector operators will often be found in this
chapter.

\begin{script}[Derivative of $f:\R\rightarrow\R^{m}$ at $u$]
\begin{smallplasm} 
DEF D1m (f::IsSeqOf:IsFun)(u::IsReal) = \+\\
%(AA:*  $\sim$  DISTR): 
%     < AA:K:((CONS  $\sim$  AA:D11):f:u), S1>;
  S1 scalarVectProd AA:K:((CONS $\sim$ AA:D11): f: u);
\end{smallplasm} 
\label{script:5:D1m}
\end{script}

In the following examples, according to Script~\ref{script:5:D}, we use the
generalized operator \texttt{D} instead of the specialized operator
\texttt{D1m}.

Notice also that in the remainder of this chapter we usually implement
 in \pl\ a curve
$\v{\phi} = (\phi_{1}, \ldots, \phi_{m})$ as the \emph{sequence} 
\begin{smallplasm}
< $\phi_{1}$, \ldots, $\phi_{m}$ >
\end{smallplasm}
of its coordinate functions, and not as the
vector-valued function 
\begin{smallplasm}
[ $\phi_{1}$, \ldots, $\phi_{m}$ ]
\end{smallplasm}
given by their \texttt{CONS}.  Such implementation choice allows for
easier access and manipulation of the coordinate functions, and is a
consequence of the discussion in Example~\ref{ex:5:coordinatemaps}.


\begin{example}[Circular curve]
    
The circular curve of unit radius centered at the origin may be given
as a map 
\[
\texttt{circlemap} :\R\rightarrow\R^{2} ;\ u\mapsto (\cos u, \sin u), 
\qquad u\in [-\pi,\pi]
\]
The derivative $D \texttt{circlemap}(u)$ at $u$ is a linear map.  In
Figure~\ref{figure:5:ex2}a we show both the image curve
$\texttt{circlemap}[-\pi,\pi]$ and the line segment $D
\texttt{circlemap}({\pi\over 3})[-{1\over 2},{1\over 2}]$.  Notice that
the point $D \texttt{circlemap}({\pi\over 3})(0)$ clearly coincides with
$(0,0)$.
    
\begin{script}[Derivative at a circle point]
\begin{smallplasm} 
DEF circlemap \index{Library functions!Shapes library!{\tt  circlemap}} = < COS, SIN >;\\[0.3cm]

STRUCT:<\+\\
  MAP:(CONS:circlemap $\sim$ s1):(interval:<-:PI,PI>:60),\\
  T:<1,2>:(CONS:circlemap:(PI/3)),\\
  MAP:(CONS:(D:circlemap:(PI/3))):(interval:<-0.5,0.5>:1) >;
\end{smallplasm} 
\label{script:5:circle}
\end{script}

The last expression of Script~\ref{script:5:circle} produces
Figure~\ref{figure:5:ex2}b, where the set $D
\texttt{circlemap}({\pi\over 3})[-{1\over 2},{1\over 2}]$ has been
translated to the tangency point $\texttt{circlemap}({\pi\over 3})$. 
The part of tangent line there shown is exactly the point set
\[
\texttt{circlemap}({\pi\over 3}) + D
\texttt{circlemap}({\pi\over 3})[-{1\over 2},{1\over 2}].
\]
where, as usual, $f(A)$ stands for $\{ f(a)\ |\ a\in A \}$, and $x + B$
stands for $\{x + b\ |\ b\in~B\}$.


\begin{figure}[htb]
\centering
\epsfig{file=\figsdir{cap05}ex21.eps,width=0.2\linewidth}
\hspace{2.5cm}
\epsfig{file=\figsdir{cap05}ex22.eps,width=0.2\linewidth}
\caption{(a) Derivative of \texttt{circlemap} map $\R\rightarrow\R^{2}$ at
$\pi\over 3$ (b) Corresponding affine map
\label{figure:5:ex2}}
\end{figure}

\end{example}


\begin{example}[Circular helix]
    
The \emph{circular helix} curve of radius $r$,
pitch $h$ and $n$ number of $2\pi$ turns can be given as the map
\[
\texttt{helix}(r,h,n) : \R\rightarrow\R^{3} :
u \mapsto (r \cos u, r \sin u, {h\over 2\pi} u ), 
\qquad u\in[0,2\pi n].
\]

The corresponding variable-free formulation is
\[
\texttt{helix}(r,h,n) = \vet{\underline{r} \cos & \underline{r} \sin &
\underline{({h \over 2\pi})} \mbox{id}}
\]

The collection of helix curves with axis on $x=y=0$, starting point on
the $x$-axis, and prescribed radius, height and number of turns, may hence be 
generated by the \pl\ function
\[
\texttt{helix} : \R^{+}\times\R^{+}\times \R^{+} \rightarrow
(\R\rightarrow\R^{3})
\]
which is implemented in Script~\ref{script:5:D1mb} as a sequence of
three coordinate functions.
    
\begin{script}[Derivative at a helix point]
\begin{smallplasm}   
    DEF helix (r,h,n::IsRealPos) = 
  < K:r*COS, K:r*SIN, K:(h/(2*PI))*ID >;\\[0.3cm]

STRUCT:<\+\\
  MAP:(CONS:(helix:<1,0.5,4>) $\sim$ s1):(interval:<0,8*PI>:180),\\
  T:<1,2,3>:(CONS:(helix:<1,0.5,4>):(2*PI/3)),\\
  MAP:(CONS:(D:(helix:<1,0.5,4>):(2*PI/3))):(interval:<-0.5,0.5>:1)\-\\
>;
\end{smallplasm} 
\label{script:5:D1mb}
\end{script}
The \texttt{STRUCT} expression gives both the helix image of the
interval $[0,8\pi]$ and the image of the tangent map in ${2\over
3}\pi$ on the interval $[-{1\over 2}, {1\over 2}]$.  To be precise,
the \texttt{STRUCT} expression of Script~\ref{script:5:D1mb} produces
(a simplicial approximation of) the sets
\begin{eqnarray*} 
&&\texttt{helix}(r,h,n)[0,8\pi] \qquad\mbox{and}
\\
&&\texttt{helix}(r,h,n)({2\over 3}\pi) + D
\texttt{helix}(r,h,n)({2\over 3}\pi)[-{1\over 2},{1\over 2}]
\end{eqnarray*}
with $r=1$, $h=0.5$ and $n=4$.  Both such sets are shown in
Figure~\ref{figure:5:ex3}.

\begin{figure}[htb]
\centering
\epsfig{file=\figsdir{cap05}ex3.eps,width=0.6\linewidth}
\caption{Image of the tangent map to the \texttt{helix} curve
% $\v{\sigma}:\R\rightarrow\R^{3}$ 
at the point ${2\over 3}\pi$
\label{figure:5:ex3}}
\end{figure}

\end{example}


\subsection{Real-valued maps of several real variables}\index{Real-valued maps!of several real variables}

In this case we have $E = \R^{n}$ and $F = \R$, where
$\phi:\R^{n}\rightarrow\R$ is a real-valued function of $n$ real
variables.  The derivative $D{\phi}(\p{a})$ at $\v{a}$ (if it exists),
is a \emph{linear functional} $\R^{n} \rightarrow \R$, called the
\emph{gradient} of ${\phi}$ at $\p{a}$, and often denoted by $\nabla
\phi(\p{a})$.

In order to compute $D{\phi}(\p{a})$, notice that each $\v{h}\in\R^{n}$
can be written, using the standard basis $( \v{e}_{1}, \ldots,
\v{e}_{n} )$, as $ \v{h} = h_{1}\v{e}_{1} + \cdots + h_{n}\v{e}_{n}$,
so that, by linearity
\[
D{\phi}(\p{a})(\v{h}) = h_{1}D{f}(\p{a})(\v{e}_{1}) + \cdots +
h_{n}D{f}(\p{a})(\v{e}_{n}).
\]

Each $D{f}(\p{a})(\v{e}_{i})$ is the limit of the ratio of the
function difference in two close points $\p{a}-dx_{i}\v{e}_{i}$ and
$\p{a}+dx_{i}\v{e}_{i}$, over their distance $2\,dx_{i}$, but this
limit of the ``incremental ratio" is exactly the derivative of $f$ at
$\p{a}$ as a function of a single real variable, usually known as the
\emph{partial derivative} ${\partial \phi \over \partial x_{i}}(\v{a})
\in \lin(\R^{n};\R)$, so that we can write:
\[
D{\phi}(\p{a})(\v{h}) = h_{1}{\partial \phi \over \partial x_{1}}(\v{a}) +
\cdots + h_{n}{\partial \phi \over \partial x_{n}}(\p{a}) 
= \nabla\phi(\p{a}) \cdot \v{h}.
\]

\paragraph{Implementation}\index{Implementation!derivative maps!real-valued, of several real variables}

To compute a partial derivative \texttt{Dp} of $\phi$ at $\p{a}$ in
the $i$-th coordinate direction, it is necessary to compute the
function difference in two points \texttt{a1} and \texttt{a2} which
differ only for a suitably ``small" amount $2\,dx$ of the $i$-th
coordinate, if we choose a numerical scheme based on central
differences.

\begin{script}[Partial derivative of $f:\R^{n}\rightarrow\R$ at $a$]
\begin{smallplasm} 
DEF Dp 
(i::isIntPos)(f::IsFun)(a::IsPoint)(h::IsVect) = \+\\
  (f:a2 - f:a1) / (2*dx) \-\\
WHERE \+\\
  a2 = CAT:<seq1, <SEL:i:a + dx>, seq2> , \\
  a1 = CAT:<seq1, <SEL:i:a - dx>, seq2> , \\
  seq1 = AS:SEL:(1..(i - 1)):a , \\
  seq2 = AS:SEL:((i + 1)..n):a , \\
  n = LEN:a ,\\
  dx = 1E-5\-\\
END;
\end{smallplasm} 
\label{script:5:Dp}
\end{script}

The \emph{gradient} at $\p{a}$ of a function
$\phi:\R^{n}\rightarrow\R$ is the column vector of the partial
derivatives of $\phi$ at $\p{a}$.  It is easily implemented in
Script~\ref{script:5:grad}.

\begin{script}[Gradient of $\phi:\R^{n}\rightarrow\R$ at $a$]
\begin{smallplasm} 
DEF grad \index{Library functions!Derivatives library!{\tt  Grad}} (f::IsFun)(a::IsPoint) = CONS:(DpVect:f):a\\
WHERE\+\\
  DpVect = (CONS  $\sim$  AA:Dp  $\sim$  INTSTO  $\sim$  LEN):a\-\\
END;\\[0.3cm]

DEF f = SIN $\sim$ S1 * SIN $\sim$ S2;\\[0.3cm]
grad:f 
$\equiv$ An-Anonymous-Function \\

grad:f:<PI/3,PI/-2>
$\equiv$ < An-Anonymous-Function, An-Anonymous-Function >\\

CONS:(grad:f:<PI/3,PI/-2>):<1,1>
$\equiv$ <-0.4999999999921733, 0.0>
\end{smallplasm} 
\label{script:5:grad}
\end{script}

Some examples of use of the \texttt{grad} operator are given in 
Script~\ref{script:5:grad}.  They may help to understand the semantics of the
\texttt{grad} implementation.

It may be also useful to show the variable-free representation of the
map generated when using the \texttt{Dn1} operator given in
Script~\ref{script:5:Dn1}.  Notice first of all, that such a function,
like most \pl\ programs, is dimension-independent and works for any
dimension $n$ of the domain space.

So, we have
\[
\texttt{Dn1}(\phi)(\p{a}) := 
 {\partial \phi \over \partial x_{1}}(\v{a})  \sigma({1}) +
\cdots + {\partial \phi \over \partial x_{n}}(\p{a}) \sigma({n}) ,
\]
where $\sigma({1}),\ldots,\sigma({n})$ are the selector functions
defined in Section~\ref{sec:5:free}, and 
\[
\texttt{Dn1}(\phi)(\p{a})(\p{u}) = 
 {\partial \phi \over \partial x_{1}}(\v{a}) u_{1} +
\cdots + {\partial \phi \over \partial x_{n}}(\p{a}) u_{n}  = 
\nabla \phi (\p{a}) \cdot \v{u}.
\]


\begin{script}[Derivative of $\phi:\R^{n}\rightarrow\R$ at $\p{a}$]
\begin{smallplasm} 
DEF Dn1 (f::IsFun)(a::IsPoint) = \+\\
  (InnerProd  $\sim$  [AA:SEL  $\sim$  INTSTO  
  $\sim$  LEN, grad:f]):a ;
\end{smallplasm} 
\label{script:5:Dn1}
\end{script}

The reader should notice that the \pl\ expression 
\begin{smallplasm}
  (AA:SEL $\sim$ INTSTO $\sim$ LEN):a
\end{smallplasm}
returns the sequence \texttt{< $\sigma({1})$, \ldots, $\sigma({n})$ >}
of selector functions, as well as the expression\ \texttt{grad:f:a}\
returns the sequence of partial derivatives of \texttt{f} at
\texttt{a}.  Also notice that the binary \texttt{InnerProd} operator,
defined in Script~\ref{script:UnitVect}, works with pairs of ordered
$d$-tuples of either scalar numbers or scalar functions.

As always, we will use in the remainder of this chapter the generalized
operator \texttt{D} of Script~\ref{script:5:D}, instead of the
specialized derivation operator \texttt{Dn1}.


\paragraph{Directional derivative}\index{Directional!derivative}

The weaker concept of \emph{directional} (or \emph{Gateaux})
\emph{derivative} is defined as follows. Let $\v{h}\in \R^{n}$ be a 
unit vector and 
\[
\psi:\R\rightarrow\R;\ t\mapsto \phi(\p{a}+t\v{h}).
\]
Then the directional derivative of $\phi$ in the chosen direction $\v{h}$ is 
the scalar $\psi'(0) = D\psi(0)(1)$.  

This directional derivative may be interpreted as the
derivative along any curve passing for $\p{a}$ with $\v{h}$ as tangent
vector at $\p{a}$:
\[
D_{\v{h}}\phi(\p{a}) = \lim_{t\rightarrow 0} {{\phi(\p{a}+t\v{h})}\over t}
\]

The map $\phi$ is said to be \emph{Gateaux-differentiable} at $\p{a}$ if
$D_{\v{h}}\phi(\p{a})$ exists for all $\v{h}\in\R^{n}$. When $D\phi(\v{a})$ 
exists, 
$D_{\v{h}}\phi(\v{a})=D\phi(\v{a})(\v{h})=\nabla\phi(\p{a})\cdot\v{h}$.

\begin{example}[Graph of tangent space]

The graphs of $\sin u \sin v$, i.e. of the function $\phi =
(\sin\circ\,\sigma(1)) (\sin\circ\,\sigma(2))$ on the set $U_{1} =
[0,\pi]^{2}$, and of the affine map $(\phi + D\phi)({\pi\over
3},{\pi\over 2})$ on the set $U_{2} = [-{1\over 2},{1\over 2}]^{2}$,
is produced by Script~\ref{script:5:sinsin} and is shown in
Figure~\ref{figure:5:ex41}.

\begin{script}[Graph of tangent space]
\begin{smallplasm} 
DEF f ~= SIN $\sim$ S1 * SIN $\sim$ S2;\\
DEF U1 = interval2D:<0,0,PI,PI>:<20,20>;\\
DEF U2 = interval2D:<-0.5,-0.5,0.5,0.5>:<2,2>;\\[0.3cm]

STRUCT:< \+\\
  MAP:[S1, S2, f]:U1 CREASE (PI/2),\\
  T:<1,2,3>:([S1, S2, f]:<PI/3, PI/2>),\\
  MAP:[S1, S2, D:f:<PI/3, PI/2>]:U2 COLOR RED\-\\
>;
\end{smallplasm} 
\label{script:5:sinsin}
\end{script}

Within the above \pl\ code two rendering-oriented language constructs
are informally introduced.  The \texttt{CREASE} binary operator,
applied to a polyhedral complex and to an angle (in radiants),
annotates the first argument with a rendering property checked by the
\textsf{VRML}\index{VRML} exporter.  The resulting effect is the exporting of the
surface with ``indexed coordinates", so that a ``Gouraud's shader" can be
applied by the \textsf{VRML}\index{VRML} browser in rendering the surface. 
Analogously, the binary operator \texttt{COLOR} applies the predefined
RGB constant \texttt{RED} to its polyhedral argument.  Both the
\texttt{CREASE} and the \texttt{COLOR} operators require the loading
of the \pl\ library named ``\texttt{colors}". Such topics are  
discussed in detail in Chapter~\ref{chap.rendering}.

\begin{figure}[htb]
\centering
\epsfig{file=\figsdir{cap05}ex41.eps,width=0.48\linewidth}
\epsfig{file=\figsdir{cap05}sinsin.eps,width=0.51\linewidth,height=2.7cm}
\caption{Graphs of map $\phi$ on $[0,\pi]^{2}$ and of map $(\phi +
D\phi)({\pi\over 3},{\pi\over 2})$ on $[-1,1]^{2}$, with  $\phi =
\sin u\, \sin v$: (a) polyhedral approximation (b) smooth rendering
\label{figure:5:ex41}}
\end{figure}

In more precise terms, the \texttt{STRUCT} expresssion of
Script~\ref{script:5:sinsin} generates a simplicial approximation of
the set:
\[
(\sigma(1), \sigma(2), \phi)[0,\pi]^{2} 
\cup \left(
(\sigma(1), \sigma(2), D\phi({\pi\over 3},{\pi\over 2}))[-1,1]^{2}
+\phi({\pi\over 3},{\pi\over 2})
\right)
\]

\end{example} 



\subsection{Vector-valued maps of several real variables}\index{Vector!-valued maps of several real variables}

In this case we have $E = \R^{n},F = \R^{m}$ and $\T{\phi}:\R^{n}
\rightarrow \R^{m}$.  The derivative $D\T{\phi}(\p{u})$ of $\T{\phi}$
at $\p{u}$ is a \emph{linear map} $\R^{n}\rightarrow\R^{m}$, if it
exists.
% , i.e.~$D\T{\phi}(\p{u})\in\R^{m}_{n}$.  

In other words, $D\T{\phi}(\p{u})$ is a matrix in $\R^{m}_{n}$,
called \emph{Jacobian matrix} of $\T{\phi}$~at~$\p{u}$.

By chosing bases $(e_{i})$ in $\R^{n}$ and $(\epsilon_{j})$ in
$\R^{m}$, with $\T{\phi} = (\phi_{1}, \phi_{2}, \ldots, \phi_{m})$ and each
$\phi_{i}: \R^{n}\rightarrow\R$, we have
\[
D\T{\phi}(\p{u}) = D\phi_{1}(\p{u})\epsilon_{1} +
D\phi_{2}(\p{u})\epsilon_{2} + \cdots + D\phi_{m}(\p{u})\epsilon_{m},
\]

The Jacobian matrix $D\T{\phi}(\p{u})$ is often represented as
\[
{\partial(\phi_{1}, \phi_{2}, \ldots, \phi_{m})
\over\partial(x_{1}, x_{2}, \ldots, x_{n})} = (\alpha_{ij}),
\quad\mbox{with}\quad \alpha_{ij} = 
{\partial \phi_{i}\over \partial x_{j}}(x) .
\]

The determinant
\[
J=\left| {\partial(\phi_{1}, \phi_{2}, \ldots, \phi_{m})
\over\partial(x_{1}, x_{2}, \ldots, x_{n})} \right|
\]
of the Jacobian matrix at $\p{u}$ is called the
\emph{Jacobian} of $\T{\phi}$ at $\p{u}$. When $\T{\phi}\in 
C^{1}(\R^{n};\R^{n})$ is a transformation of coordinates, $J(\p{u})$ 
gives the ratio of the volume elements at $\p{u}$, with $dV' = J(\p{u}) 
dV$. 


\paragraph{Implementation}\index{Implementation!derivative maps!vector-valued, of several real variables}

Once again, this derivative is quite easy to implement in \pl. 
According to the functional character of the language, the easiest
implementation of $D\T{\phi}(\p{u})$ is as the sequence
$(D\phi_{1}(\p{u}),\ldots,D\phi_{m}(\p{u}))$ of derivatives at $\p{u}$
of the coordinate functions of $\T{\phi}$. Such a derivative is denoted 
as \texttt{Dnm} in Script~\ref{script:5:Dnm}.

\begin{script}[Derivative of $f:\R^{n}\rightarrow\R^{m}$ at $a$]
\begin{smallplasm} 
DEF Dnm (f::IsSeqOf:IsFun)(u::IsPoint) = (CONS  $\sim$  AA:Dn1):f:u;
\end{smallplasm} 
\label{script:5:Dnm}
\end{script}
As is usual at this point, we will refer to this operator as
\texttt{D}, according to the generalized implementation of the
derivative operator given in Script~\ref{script:5:D}.

\begin{example}[Tangent plane at a point of a torus]
    
The parametric equations of the family of surfaces known as \emph{toruses}
can be seen as generated by a map
\[
\T{\phi}:\R^{+}\times\R^{+}\rightarrow(\R^{2}\rightarrow\R^{3}):
(r,R)\mapsto (\phi_{x}, \phi_{y}, \phi_{z}),
\]
where $r,R$ are the \emph{minor} and \emph{major} radiuses of the
surfaces, respectively, and  
\[
	\phi_{x}, \phi_{y}, \phi_{z} : U\rightarrow\R, 
\]
with
\begin{eqnarray*}
    \phi_{x} &=& (\underline{r} \cos \circ\ \sigma(1) + \underline{R}) \cos 
			\circ\ \sigma(2), \\
	\phi_{y} &=& (\underline{r} \cos \circ\ \sigma(1) + \underline{R}) \sin \circ\ \sigma(2), \\
	\phi_{z} &=&  \underline{r} \sin \circ\ \sigma(1),
\end{eqnarray*}
and
\[
	U = [0,2\pi)^{2} \subset \R^{2}.
\]

Such equations can be easily derived from the general equation of 
rotational surfaces given in Chapter~\ref{chapt:surfaces}. They are directly 
translated into the \pl\ definition given in Script~\ref{script:5:torus}.

\begin{script}[Torus]
\begin{smallplasm} 
DEF torus \index{Library functions!Shapes library!{\tt  Torus}} (r1,r2::IsReal) = < fx, fy, fz >\\
WHERE\+\\
  fx = (K:r1 * COS $\sim$   S1 + K:r2) * COS $\sim$   S2,\\
  fy = (K:r1 * COS $\sim$   S1 + K:r2) * SIN $\sim$   S2,\\
  fz = (K:r1 * SIN $\sim$   S1)\-\\
END;
\end{smallplasm} 
\label{script:5:torus}
\end{script}

Some preliminary check on the types of elementary expressions
involving the \texttt{torus} function may be useful for a deeper
understanding.  In particular, we check that
\begin{enumerate}
    \item
    the expression \texttt{torus:<1,3>} returns a triplet of
    (coordinate) functions;
    \item
    such functions, evaluated at the same 2D point return
    a 3D point;
    \item
    the derivative of \texttt{torus:<1,3>} at a point returns a triplet
    of (linear) functions, 
\end{enumerate}
as shown by the following examples:
\begin{smallplasm} 
torus:<1,3> $\equiv$ \+\\
<An-Anonymous-Function, An-Anonymous-Function, 
     An-Anonymous-Function>\-\\[0.3cm]

CONS:(torus:<1,3>):<PI/3,PI/3> $\equiv$ \+\\
<1.75, 3.031088913245535, 0.8660254037844386>\-\\[0.3cm]

D:(torus:<1,3>):<PI/3,PI/3>  $\equiv$ \+\\ 
<An-Anonymous-Function, An-Anonymous-Function, 
    An-Anonymous-Function> 
\end{smallplasm} 

\begin{figure}[htb]
\centering
\epsfig{file=\figsdir{cap05}torusman.eps,width=0.6\linewidth}
% \epsfig{file=\figsdir{cap05}toro.eps,width=0.6\linewidth}
\caption{Image of the affine map $(\T{\phi}+D\T{\phi})(\p{u})$
on the torus surface
\label{figure:5:ex51}}
\end{figure}

In Figure~\ref{figure:5:ex51} we show the image of the function
\texttt{torus:<1,3>} on the domain $[0,2\pi]^{2}$, i.e.~the torus
surface with radiuses $(1,3)$, and the image of the affine map
$\R^{2}\rightarrow\R^{3}$ of type $(\T{\phi}+D\T{\phi})(\p{u})$ on the 
domain $[-{1\over 2},{1\over 2}]^{2}$.

More formally, we can say that Figure~\ref{figure:5:ex51} gives a
picture of the set 
\[
\textstyle{ \T{\phi}[0,2\pi]^{2} \ \cup\
(\T{\phi}(\v{u})+D\T{\phi}(\p{u})[-{1\over 2},{1\over 2}]^{2} )}
\]
with $\T{\phi} = \texttt{torus:<1,3>}$ and
$\p{u}=({\pi\over3},{\pi\over3})$.

\begin{script}[Image of the affine map $(\T{\phi}(\v{u})+D\T{\phi}(\p{u}) 
\v{v}$]
\begin{smallplasm} 
STRUCT:<\+\\
  MAP:(CONS:(torus:<1,3>)):(interval2D:<0,0,2*PI,2*PI>:<24,48>) ,\\
  
  T:<1,2,3>:(CONS:(torus:<1,3>):<PI/3,PI/3>),\\
  
  MAP:(CONS:(D:(torus:<1,3>):<PI/3,PI/3>)):\+\\
    (interval2D:<-0.5,-0.5,0.5,0.5>:<2,2>) COLOR RGB:<1,0,0>  >;
\end{smallplasm} 
\label{script:5:ex51}
\end{script}
    
Actually, the implementation in Script~\ref{script:5:ex51} is based on 
the graphics approach of translating the image of the linear map 
$D\T{\phi}(\p{u})$ with translation vector $\T{\phi}(\p{u})$.
A~direct implementation of the affine mapping
\[
\v{v} \mapsto \T{\phi}(\v{u})+D\T{\phi}(\p{u}) \v{v}, 
\qquad u = \left({\pi\over 3}, {\pi\over 3}\right), 
\v{v}\in \left[-{1\over 2},{1\over 2}\right]^{2} , 
\]
should instead be written as


\begin{smallplasm} 
    MAP:(vectSum\\
    $\sim$CONS:([K$\sim$CONS:(torus:<1,3>), 
    CONS$\sim$D:(torus:<1,3>)]:<PI/3,PI/3>)):\+\\
    (interval2D:<-0.5,-0.5,0.5,0.5>:<1,1>)
\end{smallplasm} 
Notice that the greatest care was used in \texttt{CONS}ing some
sub-expressions, depending on whether they return either a single
vector function or a sequence of functions.  In fact a sequence
\emph{cannot} be applied to any argument nor composed with functions.

\end{example}


\subsection{Generalized implementation}\index{Generalized!implementation of derivative}

An implementation of a derivation operator 
\[
\texttt{D} : C^{1}(\R^{n};\R^{m}) \rightarrow \lin (\R^{n};\R^{m})
\]
at point $\p{u}\in \R^{n}$ is given in this section.  This operator
will work for arbitrary values of $n$ and $m$.  To this purpose, there
are four different cases that we need to manage, and for which we have
already given specialized operators, respectively named \texttt{D11},
\texttt{D1m}, \texttt{Dn1} and \texttt{Dnm}.

A cascaded \texttt{IF} sequence is just used to select the proper specialized 
operator, to be chosen depending on the type of input parameters \texttt{f} 
and \texttt{u}. Notice that the \texttt{TT $:=$ K:True} predicate returns 
\emph{true} for any input.

\begin{script}[Generalized derivation operator]
\begin{smallplasm} 
DEF D \index{Library functions!Derivatives library!{\tt  D}} (f::TT)(u::TT) = \+\\
  IF:< AND $\sim$ [ IsFun$\sim$S1, IsReal$\sim$S2 ], APPLY $\sim$ [D11 $\sim$ S1,S2], \\
  IF:< AND $\sim$ [ IsSeqOf:IsFun$\sim$S1, IsReal$\sim$S2 ], APPLY $\sim$ [D1m $\sim$ S1,S2], \\
  IF:< AND $\sim$ [ IsFun$\sim$S1, IsPoint$\sim$S2 ], APPLY $\sim$ [Dn1 $\sim$ S1,S2], \\
  IF:< AND $\sim$ [ IsSeqOf:IsFun$\sim$S1, IsPoint$\sim$S2 ], APPLY $\sim$ [Dnm $\sim$ S1,S2], 
  \-\\
ID >>>>:< f, u >;
\end{smallplasm} 
\label{script:5:D}
\end{script}


\section{Fields and differential operators}\index{Fields}\index{Differential operators}


A \emph{function} $\T{\phi}: {\E} \rightarrow {F}$ that maps points
of an Euclidean space ${\E}$ into a linear space ${F}$ is
called a \textbf{field}.  When ${F}$ is the set of real numbers,
or a vector or tensor space, $\T{\phi}$ is called a \emph{scalar},
\emph{vector} or \emph{tensor} \emph{field}, respectively.

\subsection{Gradient}\index{Gradient}

The \textbf{gradient} $D\phi$ of a field $\phi: {\E} \rightarrow
{F}$ is the map on ${\E}$ whose value at $x$ is the
\emph{derivative} $D\phi(x)$ of $\phi$ at $x$:

\[
D\phi : {\E} \rightarrow \lin ({\E},{F}) : x\mapsto D\phi(x)
\]
In the remainder of the chapter, we identify the Euclidean $n$-space ${\E}^{n}$ with
its linear support space $\R^{n}$, and denote the gradient of a field
$\phi$ by $\nabla \phi$.
    
% When $f$ is a scalar field on $\R$, i.e. $f:\R\rightarrow\R$, the
% gradient of $f$ evaluated at a point, i.e. $Df(x)$, is a number
% $\lambda\in\R$, which fully identifies a linear map
% \[
% \R\rightarrow\R : y \mapsto \lambda y
% \]
% 
% In this sense we have that, with $f:\R\rightarrow\R$, it is also
% \[
% Df:\R\rightarrow\R
% \]

\begin{example}[Gradient of a scalar field $\R^{m}\rightarrow \R$]
    
% \begin{script}[Graphs of some scalar fields]
% \begin{smallplasm} 
% MAP:[s1,s2,(s1*s1 - s2*s2)/K:2]:(dom_2D:<-1,-1,1,1>:<7,7>)\\
% MAP:[s1,s2,(s1*s1 + s2*s2)/K:2]:(dom_2D:<-1,-1,1,1>:<7,7>)\\
% MAP:[s1,s2,sin $\sim$ s1 * sin $\sim$ s2]:(dom_2D:<0,0,PI,PI>:<7,7>)
% \end{smallplasm} 
% \label{script:5:fields}
% \end{script}
% 
% \begin{figure}[htb]
% \centering
% \centering\epsfig{file=\figsdir{cap05}ex15a.eps,width=0.3\linewidth}
% \centering\epsfig{file=\figsdir{cap05}ex15b.eps,width=0.3\linewidth}
% \centering\epsfig{file=\figsdir{cap05}ex15c.eps,width=0.3\linewidth}
% \caption{aaaaaaaa
% \label{figure:5:ex15}}
% \end{figure}

A vector field representation of gradient $\nabla \phi$ of a scalar field
as a set of line segments is produced by the three expressions in
Script~\ref{script:5:gradients}.  In particular, each expression
aggregates the set of vectors on the vertices of the domain
discretization named \texttt{dom} and the domain itself.  The scalar
fields shown in Figure~\ref{figure:5:ex14} are, respectively:
\begin{eqnarray*}
(x,y) &\mapsto& {({x^{2} + y^{2})}/ a^{2}} 
\qquad \mbox{with\ } a =\sqrt 2, \mbox{\ and\ } \texttt{dom} = [-1,1]^{2}
\\
(x,y) &\mapsto& {({x^{2} - y^{2})}/ a^{2}}
\qquad \mbox{with\ } a =\sqrt 2, \mbox{\ and\ } \texttt{dom} = [-1,1]^{2}
\\
(x,y) &\mapsto& {\sin x \ \sin y}
\qquad \mbox{~~~with\ }  \texttt{dom} = [0,\pi]^{2}
\end{eqnarray*}

    
\begin{script}[Vector field representation of the gradient of a scalar field]
\begin{smallplasm} 
(STRUCT $\sim$ [GradMap,s2]):<(s1*s1 + s2*s2)/K:2, interval2D:<-1,-1,1,1>:<7,7>>\\
(STRUCT $\sim$ [GradMap,s2]):<(s1*s1 - s2*s2)/K:2, interval2D:<-1,-1,1,1>:<7,7>>\\
(STRUCT $\sim$ [GradMap,s2]):<sin $\sim$ s1 * sin $\sim$ s2, interval2D:<0,0,PI,PI>:<7,7>>
\end{smallplasm} 
\label{script:5:gradients}
\end{script}

The first two fields correspond to the explicit representation of
quadrics surfaces called \emph{elliptic} and \emph{hyperbolic
paraboloid}, respectively.  The function \texttt{interval2D}, to generate a
cell decomposition of the Cartesian product of two line intervals, was
given in Script~\ref{script:5:toolbox}.

\begin{figure}[htb]
\centering
\centering\epsfig{file=\figsdir{cap05}ex14b.eps,width=0.3\linewidth}
\centering\epsfig{file=\figsdir{cap05}ex14a.eps,width=0.3\linewidth}
\centering\epsfig{file=\figsdir{cap05}ex14c.eps,width=0.3\linewidth}
\caption{Graph of gradient of fields: (a) $\phi_{1}(x,y)={({x^{2} + y^{2})}/ 
a^{2}} $ (b)~$\phi_{2}(x,y)={({x^{2} - y^{2})}/ a^{2}} $  
(c) $\phi_{3}(x,y)={\sin x \ \sin y}$
\label{figure:5:ex14}}
\end{figure}

\end{example}

\paragraph{Implementation}\index{Implementation!gradient of a scalar field}

As we already know, if $\phi$ is a scalar field
$\E^{n}\rightarrow\R$, then $D\phi$ --- in the following denoted as
$\nabla \phi$ --- can be seen as the vector field such that for each 
$\p{u}\in\R^{n}$
\[
D\phi(\p{x})(\p{u}) = \nabla \phi(\p{x}) \cdot \p{u}.
\]
So, in Script~\ref{script:5:GradMap} we give a function
\texttt{GradMap} to generate a map of such a vector field for every
scalar function \texttt{f} defined on an arbitary (polyhedral) domain
\texttt{dom}.  In particular, an expression like
\texttt{GradMap:< f, dom >} will generate a set of line segments attached
to vertices of the cell decomposition of \texttt{dom}, where each
segment at point $\p{x}$ is a scaled image of the vector
\[
\nabla f(\p{x}) = \vet{ {{\partial f}\over{\partial x_{1}}}(\p{x})(1) &
\cdots & {{\partial f}\over{\partial x_{n}}}(\p{x})(1) },
\]
translated at $\p{x}$.  For this purpose a function \texttt{segment}
is given in Script~\ref{script:5:GradMap}, that generates the image of
vector $\texttt{b}-\texttt{a}$ applied to $\texttt{a}$, and scaled
with coefficient \texttt{sx}.  The segment function makes use of the
\texttt{polyline} primitive defined in Script~\ref{script:7-polyline}.

\begin{script}[Graph of gradient of a scalar field]
\begin{smallplasm} 
DEF Gradient \index{Library functions!Derivatives library!{\tt  Gradient}} (f::IsFun)(x::IsPoint) = 
CONS:(grad:f:x):(\#:(LEN:x):1);\\[0.3cm]

DEF Segment \index{Library functions!Shapes library!{\tt  Segment}} (sx::IsReal)(a,b::IsPoint) = \+\\
  (T:ind:a  $\sim$  S:ind:(\#:n:sx)  $\sim$  
  T:ind:(AA:-:a) $\sim$ polyline):<a,b>\-\\
WHERE \+\\
  n   = LEN:a, ind = INTSTO:n \-\\
END;\\[0.3cm]

DEF GradMap (f::IsFun; dom::IsPol) = (STRUCT\+\\
   $\sim$ AA:(Segment:0.35 $\sim$ [ID,vectSum $\sim$ [ID,Gradient:f]])\\
   $\sim$ S1 $\sim$ UKPOL):dom
\end{smallplasm} 
\label{script:5:GradMap}
\end{script}

The simple algorithm used in implementing the \texttt{GradMap}
function can be described as follows.  The \texttt{vectSum} function
was given in Script~\ref{script:2:sum}.
\begin{enumerate}
\item
Compute the vertices of the cells of the polyhedral decomposition of 
\texttt{dom}, by ``unpacking" its data structure and extracting their 
set with the \texttt{S1} selector.  
\item
Apply to all such vertices the function
\begin{smallplasm} 
segment:0.35 $\sim$ [ID, vectSum $\sim$ [ID, Gradient:f]]
\end{smallplasm} 
which generates a sequence of segments --- scaled with coefficient
$0.35$ --- with first point in each cell vertex.
\item
Aggregate all such segments in the output polyhedral complex.
\end{enumerate}

    
\paragraph{Gradient of a vector field}
\index{Gradient!of a vector field}

The implementation of Script~\ref{script:5:GradMap} works only for
scalar fields.  The gradient of a \emph{vector field}
$\T{\phi}:\R^{2}\rightarrow\R^{2}$ would return a tensor (represented
by the $2\times 2$ Jacobian matrix) at each point.  A graphical
representation of the output tensor field would be much harder to
devise.  Anyway, a possible implementation of a function, which
returns the gradient of a vector field in a point as the Jacobian
matrix in that point, is given by function \texttt{Jacobian} in
Script~\ref{script:5:Jacobian}.

As a check on the implementation, we compute directly $\nabla\T{\phi}(0.25,
0.3)$, with
\[
\T{\phi}(x,y)=\left({x^{2}-y^{2}\over 2}, {x^{2}+y^{2}\over 2}\right).
\]
In this case we have 
\[
\nabla \T{\phi}(x,y)
= \mat{
x  &  -y \\
x  &  y
}, \quad\mbox{and}\quad
\nabla \T{\phi}(0.25, 0.3)
= \mat{
0.25  &  -0.3 \\
0.25  &  0.3
} .
\]

As we are now used to doing, the field to be differentiated must be written 
using the variable-free notation
\[
\T{\phi} = \left(
{\sigma(1)^{2} - \sigma(2)^{2} \over \underline{2}},
{\sigma(1)^{2} + \sigma(2)^{2} \over \underline{2}}
\right).
\]

\begin{script}[Jacobian of a vector field]
\begin{smallplasm} 
DEF Jacobian \index{Library functions!Derivatives library!{\tt  Jacobian}} (f::IsSeqOf:IsFun)(x::IsPoint) = %\+\\
  CONS:(AA:Gradient:f):x;\\[0.3cm]

Jacobian:< (s1*s1 - s2*s2)/K:2, (s1*s1 + s2*s2)/K:2 >:<0.25,0.3> $\equiv$\+\\
<<0.24999999999995512, -0.2999999999999531>, \\
~<0.24999999999997247, 0.2999999999998837>> ;
\end{smallplasm} 
\label{script:5:Jacobian}
\end{script}


\subsection{Divergence and Laplacian}\index{Divergence!and Laplacian}

The divergence and Laplacian operators are defined for vector and 
scalar fields, respectively, and are both functions of the gradient.

The \emph{divergence} of a vector field $\v{\phi}$ is defined as the
\emph{trace of the gradient}:
\[
\mbox{div\ } \v{\phi} := \mbox{tr }\nabla\v{\phi}
\]
Notice that the gradient of a vector field is a tensor field, whereas 
the divergence of a vector field is a scalar field.

The \emph{Laplacian} $\Delta\phi$ of a scalar field $\phi$ is defined as the
\emph{divergence of its gradient}:
\[
\Delta\phi := \mbox{div\ }\nabla \phi = \mbox{tr }\nabla\nabla \phi = \mbox{tr
}\nabla^{2} \phi
\]
Accordingly with the above remark, notice that the Laplacian of a
scalar field is a scalar field.

\subsection{Curl}\index{Curl}

Let $\v{\phi}$ be a smooth vector field.  The \emph{curl} of $\v{\phi}$, denoted
$\curl \v{\phi}$, is the unique vector field such that
\[
(\curl \v{\phi}) \times \v{v} = (\nabla\v{\phi} - \nabla\v{\phi}^{T}) \v{v}
\]
for every vector $\v{v}$.  Unlike gradient and divergence, the usual
setting of curl can be given only in $\E^{3}$.  A more general
definition of curl requires exterior calculus.

Let us remember that (a) a tensor $\T{W}$ is said to be \emph{skew} if  
$\T{W} = -\T{W}^{T}$, and (b) there is a one-to-one correspondence 
between skew tensors and vectors. In fact, for every skew tensor $\T{W}$ 
there is one and only one vector $\v{w}$ such that 
\[
\v{w} \times \v{v} = \T{W}\v{v}
\]
for every vector $\v{v}$.  The one-dimensional space spanned by
$\v{w}$ is called \emph{axis} of the tensor $\T{W}$.  It coincides
with the null space of $\T{W}$, i.e. with the subset of vectors that
$\T{W}$ maps to $\v{0}$.

Clearly enough, \curl \v{\phi} is the axis of 
$\nabla\v{\phi} - \nabla\v{\phi}^{T}$, which is a skew tensor field.



% \subsection{Flows}\index{Flows}


\subsection{Gradient, divergence and curl of a 3D field}
\index{Gradient!of a 3D field}
\index{Divergence!of a 3D field}
\index{Curl!of a 3D field}

In elementary vector calculus in $\E^{3}$ it is customary to use the operator
\[
\nabla = 
{\partial\over \partial x}\v{e}_{1} +
{\partial\over \partial y}\v{e}_{2} +
{\partial\over \partial z}\v{e}_{3} ,
\]
referred to as either \emph{nabla} or \emph{DEL}, as a sort of vector 
which follows the standard rules of vector calculus. Using 
this operator it is possible to give three simple and well-known formulas for
gradient, divergence and curl, respectively. 

\paragraph{Gradient}\index{Gradient}

If $\phi$ is a differentiable scalar field, then the gradient of $\phi$ is 
defined~by
\[
\mbox{grad}\ \phi = \nabla \phi = 
{\partial \phi\over \partial x}\v{e}_{1} +
{\partial \phi\over \partial y}\v{e}_{2} +
{\partial \phi\over \partial z}\v{e}_{3} 
\]

\paragraph{Divergence}  

If $\v{\phi} = \phi_{1}\v{e}_{1} + \phi_{2}\v{e}_{2} +
\phi_{3}\v{e}_{3}$ is a differentiable vector field, then the
\emph{divergence} of $\v{\phi}$ is given by
\[
\mbox{div}\ \v{\phi} =
\nabla \cdot \v{\phi} 
% = \left(
% {\partial\over \partial x}\v{e}_{1} +
% {\partial\over \partial y}\v{e}_{2} +
% {\partial\over \partial z}\v{e}_{3} \right) \cdot
% (
% f_{1}\v{e}_{1} +
% f_{2}\v{e}_{2} +
% f_{3}\v{e}_{3} )
=
{\partial \phi_{1}\over \partial x} +
{\partial \phi_{2}\over \partial y} +
{\partial \phi_{3}\over \partial z} 
\]

\paragraph{Curl}\index{Curl}

If $\v{\phi}$ is a
differentiable vector field, then the \emph{curl} of $\v{\phi}$ is defined by
\[
\mbox{curl}\ \v{\phi} =
\nabla \times \v{\phi} = \det \mat{
\v{e}_{1} & \v{e}_{2} & \v{e}_{3} \\[0.2cm]
{\partial \over \partial x} & {\partial \over \partial y} & {\partial 
o\over \partial z}\\[0.2cm]
\phi_{1} & \phi_{2} & \phi_{3}
}
\]
\[
= 
\left({\partial \phi_{3}\over \partial y} - {\partial \phi_{2}\over \partial z}\right)\v{e}_{1} + 
\left({\partial \phi_{1}\over \partial z} - {\partial \phi_{3}\over \partial x}\right)\v{e}_{2} + 
\left({\partial \phi_{2}\over \partial x} - {\partial \phi_{1}\over \partial y}\right)\v{e}_{3} .
\]

\paragraph{Laplacian}  

If $\phi$ is a differentiable scalar field, then the \emph{Laplacian} 
of $\phi$ is given~by
\[
\Delta \phi =
\nabla \cdot (\nabla \phi) = {\partial^{2} \phi\over \partial x^{2}} +
{\partial^{2} \phi\over \partial y^{2}} +
{\partial^{2} \phi\over \partial z^{2}}  = \nabla^{2} \phi
\]
where $\nabla^{2}$ is called the \emph{Laplacian operator}.

\paragraph{Properties}\index{Properties}

Some well-known statements relating gradient, divergence and curl say
that, for every differentiable vector field $\v{\phi}$ and 
scalar field $\phi$:
\begin{enumerate}

\item the curl of the gradient is zero: $ (\mbox{curl} \circ
\mbox{grad})\ \phi = \nabla \times (\nabla \phi) = \v{0}$;

\item a vector field is (locally) a gradient if and only if its curl vanishes;

\item the divergence of curl is zero: $ (\mbox{div}
\circ \mbox{curl})\ \v{\phi} = \nabla \cdot (\nabla \times \v{\phi}) =
0$;

\item a vector field is (locally) a curl if and only if its divergence
vanishes.
\end{enumerate}


\paragraph{Implementation}\index{Implementation!divergence and curl}

According to its definition, the \texttt{Divergence} of a
vector field $\v{\phi}:\R^{n}\rightarrow\R^{n}$ is implemented in
Script~\ref{script:5:divergence} as the \texttt{Trace} of Jacobian
matrix of $\v{\phi}$.  The \texttt{Trace} function, that returns the
sum of diagonal elements of a square matrix, was introduced in
Script~\ref{script:Trace}.

The \texttt{Curl} operator is conversely defined here only for 3D
fields, according to the formula $\nabla\times\v{\phi}$.  Remember that
\texttt{(S2$\sim$S3):M}, where \texttt{M} is a \pl\ matrix (stored by
rows), returns the second element of the third row,
i.e.~$\texttt{m}_{32}$.

\begin{script}[Divergence and Curl]
\begin{smallplasm} 
DEF Divergence \index{Library functions!Derivatives library!{\tt  Divergence}} (f::IsSeqOf:IsFun)(x::IsSeqOf:IsReal) = \+\\
  Trace:(Jacobian:f:x);\-
  \\[0.3cm]
  
DEF Curl \index{Library functions!Derivatives library!{\tt  Curl}} (f::IsSeqOf:IsFun)(x::IsSeqOf:IsReal) = \+\\
\,[ S2$\sim$S3 - S3$\sim$S2, S3$\sim$S1 - S1$\sim$S3, 
  S1$\sim$S2 - S2$\sim$S1 ]\,:(Jacobian:f:x)

\end{smallplasm} 
\label{script:5:divergence}
\end{script}


\begin{example}[Checks on field operators]
    
Some computational checks are done in Script~\ref{script:5:check},
with the aim of showing both the usage and the result of some compound
functional expressions involving curl, divergence and gradient. 

The reader should notice that fields are written either as single
functions (scalar fields), or as sequences of functions (vector
fields), and using the variable-free notation.  In particular,
\texttt{f1} and \texttt{f2} in Script~\ref{script:5:check} are two
scalar fields in $\R^{3}$, and \texttt{g} is a vector field
$\R^{3}\rightarrow\R^{3}$.

According to our expectations, we can see that $ (\mbox{curl} \circ
\mbox{grad})\ \phi = \v{0}$ and that $ (\mbox{div} \circ \mbox{curl})\
\v{\phi} = 0$.  Those checks are done in a single
point, but it is reasonable to expect that they give the same result
everywhere, as the reader may check.

Notice also the inefficiency involved in extracting the coordinate
functions of the vector functions returned by expressions
\texttt{Gradient:f1}, \texttt{Gradient:f2} and \texttt{Curl:g}.  At
present, there is no way in \pl\ to deal efficiently with this
kind of extraction.

\begin{script}[Some computational checks]
\begin{smallplasm} 
DEF f1 = ( s1*s1 + s2*s2 )/K:2;\\
DEF f2 = ( s1*s1 + sin $\sim$ s2 - s1*s3 )/ K:2;\\
DEF g ~= < sin $\sim$ s1, cos $\sim$ s2, s1*s3 >;\\[0.3cm]

Gradient:f1:<0.5,0.1> $\equiv$ 
<0.49999999999994493, 0.10000000000010001>\\[0.3cm]

Curl:< s1 $\sim$ Gradient:f1, s2 $\sim$ Gradient:f1, s3 $\sim$ Gradient:f1 
>:<0.9,8,0>\+\\
$\equiv$ <0.0, 0.0, 0.0>\-\\
Curl:< s1 $\sim$ Gradient:f2, s2 $\sim$ Gradient:f2, s3 $\sim$ 
Gradient:f2 >:<1,0.5,1>\+\\
$\equiv$ <0.0, 0.0, 0.0>\-\\[0.3cm]

Divergence:< s1 $\sim$ Curl:g, s2 $\sim$ Curl:g, s3 $\sim$ Curl:g >:<0.5,110.5,1>
$\equiv$ 0.0\\
Divergence:< s1 $\sim$ Curl:g, s2 $\sim$ Curl:g, s3 $\sim$ Curl:g >:<-8.5,0.5,-3>
$\equiv$ 0.0
\end{smallplasm} 
\label{script:5:check}
\end{script}

Actually such a solution would exist, but it requires us to completely
renounce the $\lambda$-style of writing \pl\ functions, and its
somewhat static typing, and to adopt systematically a \emph{pure} \fl\ 
style, i.e.~\emph{without} formal parameters. 
Doing so, we would be able to write an expression like
\begin{smallplasm} 
Divergence:([s1,s2,s3] $\sim$ Curl:g):<0.5,110.5,1>
\end{smallplasm} 
This is not possible with the definition of \texttt{Divergence}
operator given in Script~\ref{script:5:divergence}, because the
interpreter would expect to apply it to a \emph{sequence} of functions,
whereas \texttt{[s1,s2,s3] $\sim$ Curl:g} is a single function.

Actually, there are some drawbacks in using only the pure \fl\ style,
including less self-documented and more intricate programs, and lack
of any type checking.  A carefully chosen compromise between
$\lambda$-style and pure \fl\ style is often preferable.

\label{example:5:checks}
\end{example}



\section{Differentiable manifolds}\index{Differentiable manifolds}

In some areas of geometric modeling, and in particular in solid
modeling, the terms ``manifold" and ``differentiable manifold" are
used quite frequently.  A quick introduction to this concept may
therefore be needed, and is given here.  A manifold is a set where it 
makes sense to introduce coordinates, at least locally. Such 
coordinates locally behave like affine coordinates.
Useful references are~Crampin
and Pirani~\cite{crampin:86}, Jones, Gray and Hutton~\cite{jones:87},
and Nakahara~\cite{nakahara:90}.


\subsection{Charts and atlases}\index{Chart}\index{Atlas}

The main concepts here are related to systems of coordinates, and are
quite easy to understand.  For several sets, for example the surface
$S_{2}$ of the Earth, only one system of coordinates is not
sufficient, because a single bijection between the $S_{2}$ points and
$2$-tuples of coordinates does not exist.  Therefore the goal is 
to introduce coordinate systems on point \emph{subsets} such that:
\begin{enumerate}
    \item
    nearby points have nearby coordinates; 
    \item
    every point has unique coordinates in each system that contains
    it;
    \item
    when two coordinate systems overlap, they must be related in a
    smooth way.
\end{enumerate}
The last requirements guarantee that differentiable functions in one
system are differentiable also in the other.

\paragraph{}\index{}

A \emph{chart} for a set $M$ is a pair $(U,\v{\phi})$, with $U\subset
M$ and $\v\phi$ a bijective function from $U$ to an open set in
$\R^{n}$.  The subset $U$ is called the \emph{coordinate
neighborhood}, while $\v{\phi} = (\phi_{1},\ldots,\phi_{n})$ is the
\emph{coordinate map}, where $\phi_{i}: U\rightarrow \R$ ($1\leq i\leq
n$) are the \emph{coordinate functions}.

Two charts $(U,\v\phi)$ and $(V,\v\psi)$ for the set $M$ are said to be
\emph{compatible} if, provided that $U \cap V \not= \emptyset$, then:

\begin{enumerate}
    \item the sets $\v\phi(U \cap V)$ and $\v\psi(U \cap V)$ are open
    subset of $\R^{n}$; 
    
    \item the map $\v\phi\circ\v\psi^{-1}$ is smooth with smooth inverse,
    i.e.~is a \emph{diffeomorphism}.
\end{enumerate}

An \emph{atlas} for the set $M$ is a collection $\{
(U_{i},\phi_{i}) \}$ of pairwise compatible charts such that
$\cup_{i} U_{i} = M$.


\subsection{Differentiable manifolds}\index{Differentiable manifolds}

Two atlases for the same set $M$ are \emph{equivalent} if every chart
in an atlas is compatible with every chart in the other atlas.  This
relation is reflexive, symmetric and transitive, i.e.~is an
equivalence relation.

Each equivalence class in the set of atlases of $M$ is said to be a 
\emph{differentiable structure} for $M$.
A \emph{differentiable manifold} is a pair $(M,{\cal S})$, 
where ${\cal S}$ is a differentiable structure for $M$.

The one-chart atlas $(\R^{n},\ \mbox{id}:\R^{n} \rightarrow\R^{n})$ 
determines a differentiable structure for $\R^{n}$ as a manifold. 
This one is probably the most trivial example, where a single chart 
covers the whole space, and where the coordinate map is the identity 
map.

\begin{example}[Atlas for the circle $S_{1}$]
Let us consider the one-dimensional circle $S_{1}=\{ \p{x}\in\E^{2} |\ \module{\p{x} - \p{o}}^{2}=1 \}$,
and choose two charts $(U_{1},\v\phi_{1})$ and $(U_{2},\v\phi_{2})$, with 
\[ 
U_{1} = S_{1}\backslash\{(1,0)\}, \quad \v\phi_{1} :
U_{1}\rightarrow(0,2\pi)\subset\R : (\cos\alpha, \sin\alpha)\mapsto\alpha ,
\]
and 
\[
U_{2} = S_{1}\backslash\{(-1,0)\}, \quad \v\phi_{2} :
U_{2}\rightarrow(-\pi,\pi)\subset\R : (\cos\beta, \sin\beta)\mapsto\beta .
\]
It this case it is easy to see that $U_{1}$ and $U_{2}$ are open sets
and that $\v\phi_{1}$ and $\v\phi_{2}$ are smooth.  Then, it can be
seen that
\[
\v\phi_{1}\circ\v\phi_{2}^{-1}=\left\{
\begin{array}{lcl}
    \mbox{id} & on & (0, \pi)  \\
    \\
    \mbox{id}-\underline{2\pi} & on & (\pi, 2\pi)
\end{array}
\right.
\]
is a diffeomorphism between $\v\phi_{2}(U_{1}\cap U_{2}) =
(0,2\pi)\backslash\{\pi\}$ and $\v\phi_{1}(U_{1}\cap U_{2}) =
(-\pi,\pi)\backslash\{0\}$.  The two charts are hence compatible, and
$U_{1}\cup U_{2} = S_{1}$.  Therefore we can conclude that
$\{(U_{1},\v\phi_{1}),(U_{2},\v\phi_{2})\}$ is an atlas for $S_{1}$.

\label{ex:5:manifold}
\end{example}

\paragraph{Manifold dimension}\index{Manifold!dimension}

Let $M$ be a differentiable manifold, and let ${\cal A} = 
\{(U_{i},\phi_{i})\}$ be an atlas for it. If all the sets 
$\phi_{i}(U_{i})$ are contained in the same $\R^{n}$, then $n$ is 
called the \emph{dimension} of the manifold. In other words, the dimension 
of a differentiable manifold is the (constant) number of coordinates of 
manifold points in all its charts.  This number is the same for any 
atlas of the manifold.

Not all differentiable manifolds have a dimension, since they may
contain subsets covered by charts with different numbers of
coordinates.

\paragraph{Product manifold}\index{Product!manifold}

The \emph{product} $M\times N$ of two manifolds $M$ and $N$ of
dimension $m$ and $n$ respectively, with atlases $\{(U_{i},
\phi_{i})\}$ and $\{(V_{j}, \psi_{j})\}$, is a manifold of dimension
$m+n$, whose atlas is $\{(U_{i}\times V_{j}, (\phi_{i},\psi_{j}))\}$. 

In this case if $(\p{p},\p{q})\in U_{i}\times V_{j}$, then
\begin{equation}
(\phi_{i},\psi_{j})(\p{p},\p{q}) = (\phi_{i}(\p{p}),\psi_{j}(\p{q})).
\label{eq:5:productman}
\end{equation}

\paragraph{Spheres and toruses}\index{Spheres and toruses}

The $n$-dimensional sphere $S_{n}$ is defined as the 
set $\{ \p{x}\in\E^{n+1} : \module{\p{x}-\p{o}}^{2} = 1 \}$.

The $2$-dimensional torus $T_{2}$ is the product manifold 
\[
T_{2} = S_{1} \times S_{1}.
\]
Clearly it is $T_{2}\subset\E^{4}$, since $S_{1}\subset\E^{2}$. The
product manifold called  $m$-dimensional torus
\[
T_{m} = S_{1} \times \cdots \times S_{1}, \quad\mbox{$m$ times}.
\]
with $T_{m}\subset\E^{2m}$, is an immediate generalization.


\begin{example}[Torus as Cartesian product of circles]
    
Our aim in this example is to generate a 3D projection of 
\[
T_{2} = S_{1} \times S_{1}.
\]
    
In Script~\ref{script:5:mantorus1} we generate the 3D surface with the
shape of a doughnut by translating and projecting against the subspace
$\{\p{x}\in\E^{4} | x_{4}=0\}$ the polyhedral complex \texttt{torus},
generated by Cartesian product of circle approximations
\texttt{MAP:(cc:8):d1} and \texttt{MAP:(cc:2):d1}, where $8$ and $2$ are
the radiuses, and \texttt{d1} is a partition of $[0, 2\pi]$ with $24$
segments.  The \texttt{schlegel3D} operator is given and discussed in
Section~\ref{subsec:4:Schlegel}.  The \texttt{interval:} function is
given in Script~\ref{script:5:toolbox}.

\begin{script}[Torus as product of circles (1)]
\begin{smallplasm} 
DEF cc (r::IsRealPos) = [ K:r * COS, K:r * SIN ]  $\sim$  S1;\\
DEF d1 = interval:< 0,2*PI >:24\\
DEF torus \index{Library functions!Shapes library!{\tt  Torus}} = (MAP: (cc:8): d1) * (MAP: (cc:2): d1);\\[0.3cm]

VRML:((schlegel3D:2  $\sim$  T:4:-11):torus CREASE (PI/2)):$'$out2.wrl$'$ ;
\end{smallplasm} 
\label{script:5:mantorus1}
\end{script}

Notice that both circles, with $r=8$ and $r=2$ respectively, were
produced by \texttt{MAP} operator as $\v\phi^{-1}[0,2\pi]$, with a
single chart $(U,\v\phi)$ depending on $r$, where
\[ 
U = \{\p{x}\in\E^{2} | (\p{x}-\p{o})^{2} = r^{2}\}, \quad \v\phi :
U\rightarrow[0,2\pi] : (r\cos\alpha, r\sin\alpha)\mapsto\alpha ,
\]
thus producing a singularity (a double coordinate value which results in
the same point) in $\v\phi^{-1}(0)= (r,0)$ and in
$\v\phi^{-1}(2\pi)=(r,0)$.  This kind of singularity, i.e.~double
points at the boundary of charts of closed surfaces, will be accepted
through the whole book, for the sake of implementation simplicity, in
particular because the \pl\ kernel is currently able to manage only
closed (i.e.~with boundary) polyhedral complexes.

\label{ex:5:torus2}
\end{example}
    
    
\begin{example}[Torus as manifold product of circles]
Given the charts $(U_{1},\v\phi_{1})$ and $(U_{2},\v\phi_{2})$ of two
circles $S_{1}(r_{1})$ and $S_{2}(r_{2})$ with different radiuses, we 
generate the torus 
\[
T_{2}(r_{1},r_{2}) = S_{1}(r_{1}) \times S_{1}(r_{2}),
\]
using the product chart $(U_{1}\times U_{2}, (\v\phi_{1},\v\phi_{2}))$.
    
A direct implementation of \texttt{torus} as the polyhedral
approximation of a \emph{product manifold} is hence given
in~Script~\ref{script:5:mantorus2}, where \texttt{c1c2} implements the
mapping $(\v\phi_{1},\v\phi_{2})^{-1}$ on $[0,2\pi]^{2}$ using the
\emph{product chart} $(U_{1}\times U_{2}, (\v\phi_{1},\v\phi_{2}))$
described in equation~(\ref{eq:5:productman}).  In particular,
$(\v\phi_{1},\v\phi_{2})^{-1}$ is implemented by the \texttt{c1c2}
function, whereas a cell decomposition of $[0,2\pi]^{2}$ is given by
the polyhedral complex \texttt{d1 * d1}, where \texttt{d1} is given in
Script~\ref{script:5:mantorus1}.  The simplicial decomposition needed
to take into account the double curvature of the doughnut surface is
automatically provided by the \texttt{MAP} operator.

\begin{script}[Torus as product of circles (2)]
\begin{smallplasm} 
DEF c1c2 (r1,r2::IsRealPos) = \+\\{}
[ K:r1 * COS$\sim$S1, K:r1 * SIN$\sim$S1, 
K:r2 * COS$\sim$S2, K:r2 * SIN$\sim$S2 ];\-\\
DEF torus \index{Library functions!Shapes library!{\tt  Torus}} (r1,r2::IsRealPos) = MAP: (c1c2:< 8,2 >): (d1 * d1);\\[0.3cm]

VRML:((schlegel3D:2  $\sim$  T:4:-11):(torus:<8,2>) CREASE (PI/2)):$'$out2.wrl$'$ ;
\end{smallplasm} 
\label{script:5:mantorus2}
\end{script}

The reader may try unbelievable variations of the generated 3D
object by making small variations to the parameters of the expression,
including the circle radiuses, the distance of the center of
projection from the origin, the translation and the rotation
parameters.  For example,  Figures~\ref{fig:5:toro4d}a
and~\ref{fig:5:toro4d}c are respectively generated as

\begin{smallplasm} 
DEF rotations = (COMP $\sim$ [R:<1,4>,R:<2,4>,R:<3,4>]):(PI/6);\\
VRML:((project:1):man CREASE (PI/2)):$'$out1.wrl$'$ ;\\
VRML:((project:1  $\sim$  rotations):man CREASE (PI/2)):$'$out3.wrl$'$ ,
\end{smallplasm} 


\begin{figure}[htb]
\centering\epsfig{file=\figsdir{cap05}toro4d1a.eps,width=0.30\linewidth}
\hfill
\centering\epsfig{file=\figsdir{cap05}toro4d1b.eps,width=0.32\linewidth}
\hfill
\centering\epsfig{file=\figsdir{cap05}toro4d3b.eps,width=0.32\linewidth}
\caption{Different projections on $x_{4}=0$ of the manifold
$T_{2}=S_{1}\times S_{1} \subset\E^{4}$:  (a)~standard projection
gives a cylinder (b) central projection (c) multiple rotations
followed by projection produce a double self-intersection in the
$\E^{3}$ embedding
\label{fig:5:toro4d}}
\end{figure}


Finally, let us notice that parametric equations  of $T_{2}$ in~$\E^{4}$ 
\[
    x_{1} = r_{1}\,\cos u,\quad
    x_{2} = r_{1}\,\sin u,\quad
    x_{3} = r_{2}\,\cos v,\quad
    x_{4} = r_{2}\,\sin v,
\]
where $(u,v) \in [0,2\pi]^{2}$, are simpler than
equations~(\ref{eq:2:torus}) in $\E^{3}$.
    
\label{ex:5:circles2torus}
\end{example}



% \subsection{Submanifolds}\index{Submanifolds}



\subsection{Tangent spaces and maps}\index{Tangent!spaces and maps}

A \emph{tangent vector} to $M$ at $x$ is an element of
$\T{T}_{x}\R^{n}$ of the form $(x, D\gamma(0)(1))$, where $x\in M$
and $\gamma: I\rightarrow M$ is a smooth parametrized curve with
$\gamma(0) = x$.

The \textbf{tangent space} to $M$ at $x$ is the set of all tangent
vectors to $M$ at $x$.  It is denoted by $\T{T}_{x}M$.

The tangent bundle of $M$ is the union of all its tangent spaces:
\[
\T{T}M = \bigcup_{x\in M}\T{T}_{x}M.
\]

    Let $f: M\rightarrow N$, with $M$ and $N$ submanifolds of $\R^{m}$ 
and $\R^{n}$, and $x\in M$.
A tangent map will send a small piece of curve through $x$ to a 
small piece of curve through $f(x)$.

The \emph{tangent map} of $f$ at $x$ is the map:
\[
\T{T}_{x}f : \T{T}_{x} M \rightarrow \T{T}_{f(x)} N 
\]
such that 
\[
(x,D\gamma(0)(1)) \mapsto (f(x), D(f \circ \gamma)(0)(1))
\]
for each differentiable curve $\gamma$ through $x$.  The
\textbf{tangent map} of $f$ is the map
\[
\T{T}f : \T{T}M \rightarrow \T{T}N : \T{T}_{x}M = \T{T}_{x}f,
\quad\mbox{for each\ }x\in M .
\]



\section{Derivatives of a curve}\index{Derivatives of a curve}

An intrinsic representation of a curve, in the sense that it depends
only on operations carried out on the curve itself and does not rely
on the ambient Euclidean space, is obtained by repeated application of
the derivation operator to the curve-generating function.  It is only
necessary that this function is (at least) twice differentiable for plane
curves, and (at least) three times differentiable for space curves.  Let
us assume only smooth curves.  This case is certainly the
more frequent and useful one in CAD applications.


\paragraph{Tangent}\index{Tangent}

The \emph{tangent} $\v{t}$ vector field, also denoted as $\p{c}'$, to a
curve $\p{c}:\R\rightarrow{\E}^{d}$, with $\p{c}=(c_{i})$,
is defined as the vector function $\p{c}':\R\rightarrow\R^{d}$
such that:
\[
\v{t} = D \p{c} = \p{c}' = \left(
c'_{i}\right).
\]
For each $u$ parameter value, $\p{t}(u)$ gives the tangent vector
to the curve at $\p{c}(u)$.
% , as the limit of a vector parallel to the
% corde for a very small interval between two closed points on the
% curve.

If $u=t$ is the time $t$, then $\v{v}(t) = D\v{c}(t)$ represents the
\emph{velocity} with which the point $\v{c}(t)$ describes the curve
image.  Analogously, $\v{a}(t) = D\v{v}(t) = D^{(2)}\v{c}(t)$
represents the \emph{acceleration}.

If $u=s$ is the curve length from $\p{c}(0)$, then $\v{t}(s)$ gives
the unit tangent vector at curvilinear abscissa $s$.  It is in fact
easy to see that $\module{\v{t}(s)} = 1$ for every $s$.


\paragraph{Normal}\index{Normal}

Analogously, a \emph{normal} to a twice derivable curve
$\p{c}:\R\rightarrow \E^{d}$ is defined as the vector function field:
$\p{c}^{(2)}:\R\rightarrow\R^{d}$ such that:
\[
\p{c}^{(2)} = D^{(2)} \p{c} =
(c^{(2)}_{i}).
\]
For each $u$ parameter value, $\p{c}^{(2)}(u)$ gives one of normal
vectors to the tangent vector for the corresponding point $\p{c}(u)$
of the curve.

It is important to notice that in Euclidean space a vector
function $\v{v}(t)$ of constant norm has the property that $\v{v}(t)$
and $D\v{v}(t)$ are orthogonal.



% 
% 
% Il versore normale al piano osculatore \`e detto {\it vettore binormale}
% 
% \[
% \v{B}(s) = \v{T}(s) \times \v{N}(s) 
% \]
% 
% Il modo in cui $\v{B}(s)$ varia con $s$, cio\`e il modulo di $\v{B}$'$(s)$, indica
% la velocit\`a di variazione del piano osculatore lungo la curva:
% \begin{eqnarray*}
% \v{B}$'$(s) &=& (\v{T}(s) \times \v{N}(s))$'$ \\
% &=& 
% \v{T}$'$(s) \times \v{N}(s) + \v{T}(s)
% \times \v{N}$'$(s) \\
%  &=& k(s)\v{N}(s) \times \v{N}(s) + \v{T}(s) \times \v{N}$'$(s) \\
%  &=& \v{T}(s) \times \v{N}$'$(s)  
% \end{eqnarray*}
% 
% Potremo scrivere:
% \[o
% \v{B}$'$(s) = -\tau(s) \v{N}(s)
% \]
% dove la funzione $\tau=\tau(s)$ \`e detta {\it torsione} di $\v{\alpha}(s)$
% 
% 
% Il teorema fondamentale della geometria differenziale delle curve in ${
% E}^3$ dice che una curva \`e completamente individuata dalle sue funzioni di
% curvatura e di torsione.
% 
% In particolare, siano $k(s)$ e $\tau(s)$ funzioni date su un intervallo $I$,
% con $k$ positiva e $C^1$ e $\tau$ continua. Allora esiste una curva
% $\p{\alpha}(s)$ definita su $I$ di cui $s,k,\tau$ sono rispettivamente la
% lunghezza degli archi, la curvatura e la torsione.
% 
% Inoltre, qualunque coppia di curve siffatte \`e congruente.\\
% Specificando i valori di $\alpha, \v{T}, \v{N}$ e  $\v{B}$ in un qualunque
% punto $s$ di $I$, la curva resta univocamente determinata


\paragraph{Curvature}\index{Curvature}


The {\it curvature} $\kappa$ of a twice differentiable curve
$\p{c}:\R\rightarrow{\E}^{d}: s\mapsto \p{o}+ \v{\alpha}(s)$, where $s$ is
the curve length from $\p{c}(0)$, is defined as the norm of the vector
function $D\v{t}$:
\[
\kappa:\R\rightarrow\R: s \mapsto \module{ D\v{t}(s) } =
\module{D^{(2)}\v{\alpha}(s)}.
\]
In other words, the curvature $\kappa(s)$ is the magnitude of the acceleration 
vector to the curve $\v{c}(s)$, under the condition that this one is 
parametrized by the arc length $s$.


\paragraph{Principal normal}\index{Principal normal}

If $D\v{t}(s)\not= \v{0}$ for all $s$, then the unit vector function
$\v{n}:\R\rightarrow\R^{d}$, such that
\[
\v{n} = {D\v{t} \over \module{D\v{t}}} = \kappa^{-1} {D\v{t}} ,
\]
is called \emph{principal normal} vector field.

\paragraph{Binormal}\index{Binormal}

The plane which contains both $\v{t}$ and $\v{n}$ at $u$ is called the
\emph{osculating plane} at $u$. The word comes from Latin, for the plane 
which contains the osculating circle, i.e.~the circle
``kissing" the curve.  The vector product of tangent and
normal vector is called the \emph{binormal} vector field:

\[
\v{b} = \v{t} \times \v{n}
\]

The norm of $D\v{b}$ gives the rate of variation of the osculating 
plane along the curve. If $u=s$ is the arc-length of the
curve $\v{c}$, then it is possible to write 
\[
D\v{b} = -\tau \v{n}
\]
where the function $\tau=\tau(s)$ is called \emph{torsion} at $\v{c}(s)$.

The \emph{fundamental theorem} of differential geometry of curves in
$\E^{3}$ says that a curve is completely determined by its curvature
and torsion functions within a congruence.

\paragraph{The Serret-Frenet formul\ae}\index{The Serret-Frenet formul\ae}

Let us study a space curve $\v{c}$ parametrized by the arc
length $s$, also called the \emph{natural parameter}.

The tangent $\v{t}$, the principal normal $\v{n}$ and the binormal $\v{b}$
vector functions  are related to their
derivatives by the curvature $\kappa$ and the torsion $\tau$, as shown by the
Serret-Frenet formul\ae:
\[
D\vet{\v{t}\\ \v{n}\\ \v{b}} =
\mat {
0 & \kappa & 0 \\
-\kappa & 0 & \tau \\
0 & -\tau & 0
}
\vet{\v{t}\\ \v{n}\\ \v{b}}
\]


% \begin{example}[Circular arc]
% 
% The three \texttt{MAP} expressions
% given below produce the graph of circle portion, and of tangent and
% principal normal to such a curve, that are shown in
% Figure~\ref{fig:curvature} a,b and c, respectively.
% 
% \begin{smallplasm}
%   MAP:(CONS:alpha $\sim$ s1):(interval:<0,1>:20),\\
%   MAP:(tangent:alpha $\sim$ s1):(interval:<0,1>:20),\\
%   MAP:(principalNormal:alpha $\sim$ s1):(interval:<0,1>:20)
% \end{smallplasm}
%     
% \begin{figure}[ptb]
% \centering\epsfig{file=\figsdir{cap05}curvature.eps,width=0.8\linewidth}
% \caption{Graph of portion of unit circle (first quadrant), tangent
% vector curve (second quadrant), and principal normal curve (third quadrant)
% \label{fig:curvature}}
% \end{figure}
% 
% \label{ex:curvature2}
% \end{example}


\paragraph{Center of curvature}\index{Center of curvature}


For an arbitrary curve $\p{a} \in \E^{n}$, the curvature field
$\kappa:\R\rightarrow\R$ is not constant.  Where
$\kappa(s)\not=0$, the curve is approximated by a circle of radius
$1/\kappa(s)$, which is tangent to $\p{a}$ in $\p{a}(s)$ and which is
contained in the plane generated by $\v{t}(s)$ and $\v{n}(s)$.

This circle, called \emph{osculating circle}, has the same tangent, curvature
and principal normal of the curve in~$s$.  The center of the
osculating circle is called \emph{center of curvature} of $\p{a}$ in
$\p{a}(s)$. It is a field $\p{c}: \R\rightarrow\R^{d}$ that can be 
written, using the variable-free notation, as:
\[
\p{c} = \p{a} + \kappa^{-1} \v{n}.
\]

% \paragraph{Osculating circle}\index{Osculating circle}
% 
% per una curva $\v{\alpha}$ arbitraria, la curvatura non \`e costante.
% 
% Dove $k(s)\not=0$ la curva \`e approssimata da un cerchio di raggio
% $1/k(s)$, tangente ad $\v{\alpha}$ in $\v{\alpha}(s)$ e che giace nel piano di 
% $\v{t}(s)$ e di $\v{n}(s)$
% 
% Il cerchio, detto {\it osculatore}\\
% (baciante, dal latino), ha la stessa tangente, normale principale e curvatura
% della curva~in~$s$.
% 
% Il centro del cerchio osculatore \`e detto {\it centro di curvatura} di
% $\v{\alpha}$ in $\v{\alpha}(s)$:
% 
% \[
% \p{\beta}(s) = \v{\alpha}(s) + {1\over k(s)}\v{n}(s)
% \]
% 
% 
% 
% Il centro del cerchio osculatore \`e detto {\it centro di curvatura} di
% $\v{\alpha}$ in $\v{\alpha}(s)$:
% 
% \[
% \p{\beta}(s) = \v{\alpha}(s) + {1\over k(s)}\v{n}(s)
% \]
% 

% \begin{figure}[h]
% %\def\fig:osculCircle{\scaledpicture 108mm by 62mm (osculCircle.eps scaled 800)}
% %{\fig:osculCircle}
% \centering\epsfig{file=\figsdir{cap05}osculCircle.eps,width=90mm}
% \caption{aaa}
% \end{figure}


\section{Examples}


% \subsection{Circle}\index{Circle}
% 
% \paragraph{Repeated derivatives}\index{Repeated derivatives}
% 
% More in general, we write $\p{c}^{(n)}= (c_{i}^{(n)})$, where
% \[
% c_{i}^{(n)} = \deriv{}{u} c_{i}^{(n-1)},
% \]
% to denote the $n$-th derivative of a smooth curve with respect to the
% parameter.
% 
% 
% \begin{example}[Derivatives of a curve]
% \label{ex:dercurve}
% 
% Let remember that a curve is a vector-valued function
% $\p{c}:\R\rightarrow \E^{d}$.  So, in order to compute the
% $\p{c}^{(n)}$ derivative we need: 
% \begin{enumerate}
%     \item to compute by \texttt{Curve2MapVect} the sequence of
%     coordinate functions of the input curve;
% 
%     \item to apply to all such functions the operator \texttt{DD:n};
% 
%     \item to transform the generated sequence $(c_{i}^{(n)}(u))$ into
%     a single vector-valued function by the \texttt{CONS} operator.
% \end{enumerate}
% 
% This approach is implemented by the \texttt{XDD} function defined in
% Script~\ref{script:dercurve}.  In particular, the operator
% \texttt{XDD:n} applied to a curve $\alpha$ produces the $n$-derivative
% curve $\alpha^{(n)}$.  The geometric object generated by the
% \texttt{STRUCT} expression is shown in Figure~\ref{fig:dercurve}.
% 
% \begin{script}[N-th derivative of a curve]
% \begin{smallplasm}
% DEF XDD (n::IsInt) = CONS$\sim$AA:(DD:n)$\sim$Curve2MapVect;\\[0.3cm]
% 
% DEF alpha = [ cos, sin ]$\sim$(K:(PI/2) * S1); \\
% DEF domain = (QUOTE$\sim$\#:10):(1/10);\\[0.3cm]
% 
% STRUCT:<\+\\
% MAP:(alpha):domain,\\
% MAP:(XDD:1:alpha):domain,\\
% MAP:(XDD:2:alpha):domain,\\
% MAP:(XDD:3:alpha):domain \-\\
% >;
% \end{smallplasm}
% \label{script:dercurve}
% \end{script}
% 
% \begin{figure}[ptb]
% \centering\epsfig{file=\figsdir{cap05}derCircle.eps,width=0.4\linewidth}
% \caption{Image of curve $\alpha=[\cos,\sin]\circ(\underline{\frac{\pi}{2}}
% \textrm{id})$ in first quadrant; images of derivatives curves 
% $\alpha^{(1)}$, $\alpha^{(2)}$ and $\alpha^{(3)}$ in $2^{nd}$, 
% $3^{rd}$ and  $4^{th}$ quadrant, respectively
% \label{fig:dercurve}}
% \end{figure}
% 
% \end{example}

% \subsection{helix}\index{helix}
% 

\paragraph{Helix}\index{Helix}

The \emph{circular helix}  curve in $\E^{3}$ has
vector equation $\p{s} = \p{o}+\v{\beta}(u)$, where the $u$ parameter
denotes the angle between the $\p{s}$ projection in $z=0$ and the $x$
axis.  For a helix segment we have $\v{\beta}:U\rightarrow\R^{3}$,
with $U\subset\R$, and
\begin{equation}
\v{\beta}(u) = \vet{a\cos u & a\sin u  &b u } 
\label{eq:5:helix}
\end{equation}
where $a$ is the radius of the circular projection in $z=0$ and $b$ is
the ratio \emph{pitch}$/2\pi$.  The \emph{pitch} of the helix is the
vertical distance of two curve points differing for $2\pi$ in the
parameter value.



\paragraph{Helix curvature and principal normal}\index{Helix!curvature and principal normal}
    
In Script~\ref{script:curvature} we have implemented a
\texttt{curvature} and \texttt{principalNormal} functions, by
abstracting with respect to the argument curve.  The geometric object
generated by evaluating the \texttt{STRUCT} expression is shown in
Figure~\ref{fig:5:intrinsic}.  The \texttt{helix} curve
\texttt{domain} is defined in Script~\ref{script:5:D1mb}. The 
\texttt{MKvector} function is given in Script~\ref{script:6-mkvector}.

Notice that, according to the \texttt{STRUCT} semantics discussed in 
Chapter~\ref{chapt:structures}, the translation operator is applied 
to all subsequent vector models. The intrinsic triplet there generated 
is associated to the value $\pi\over 2$ of the curve parameter.

% The geometric object generated by the \texttt{STRUCT} expression of
% Script~\ref{script:curvature} is shown in
% Figure~\ref{fig:5:intrinsic}.
    
\begin{script} [Intrinsic triplet]
\begin{smallplasm}
DEF tangent \index{Library functions!Derivatives library!{\tt  Tangent}} (curve::IsSeqOf:IsFun) = UnitVect $\sim$ CONS:(AA:D:curve);\\
DEF curvature \index{Library functions!Derivatives library!{\tt  Curvature}} (curve::IsSeqOf:IsFun) = 
VectNorm $\sim$ CONS:(AA:(D $\sim$ D):curve);\\
DEF principalNormal \index{Library functions!Derivatives library!{\tt  Principalnormal}} (curve::IsSeqOf:IsFun) = CONS:\+\\
  ((K:1 / curvature:curve) scalarVectProd (AA:(D $\sim$ D):curve)); \-\\
DEF binormal \index{Library functions!Derivatives library!{\tt  Binormal}} (curve::IsSeqOf:IsFun) = \+\\
  tangent:curve LIFT:vectProd principalNormal:curve ;\-\\[0.3cm]

DEF curve = helix:<1,0.2>;\\
STRUCT:<\+\\
  MAP:(CONS:curve  $\sim$  s1):(interval:<0,4*PI>:60),\\
  T:<1,2,3>:(CONS:curve:(PI/2)),\\
  MKvector:<0,0,0>:(tangent:curve:(PI/2)),\\
  MKvector:<0,0,0>:(principalNormal:curve:(PI/2)),\\
  MKvector:<0,0,0>:(binormal:curve:(PI/2))\-\\
>;
\end{smallplasm}
\label{script:curvature}
\end{script}

\begin{figure}[ptb]
\centering\epsfig{file=\figsdir{cap05}intrinsic.eps,width=0.4\linewidth}
\caption{Intrinsic triplet at a helix point
\label{fig:5:intrinsic}}
\end{figure}

\paragraph{Helix parametrized by arc length}\index{Helix!parametrized by arc length}

Let us reparametrize the helix (\ref{eq:5:helix}) as a function of
the arc length $s$.  So we have:
\[
D\v{\beta}(u) = \vet{-a\sin u & a\cos u  &b }
\]
\[
{\deriv s u} = \module{D\v{\beta}(u)} = \sqrt{a^2 + b^2}
\]
\[
s = \int_0^s ds  = \int_0^u \sqrt{a^2 + b^2}\ du = u \sqrt{a^2 + b^2}
\]
and hence we can write $\p{s} = \p{o}+\v{\alpha}(s)$, with $s\in [0,
L]$ and \def\aa{{s\over\sqrt{a^2 + b^2}}}
\[
\v{\alpha}(s) = \vet{a\cos\aa & a\sin\aa  &b \aa }
\]

Two helixes of the same length $L=6\pi$ generated by $\v{\alpha}(s)$ 
are shown in Figure~\ref{fig:5:eqhelixs}.
\begin{figure}[h]
\centering\epsfig{file=\figsdir{cap05}ascissaCurv.eps,width=0.6\linewidth}
\caption{Two helixes of the same length and different radius
\label{fig:5:eqhelixs}}
\end{figure}


\paragraph{Osculating circle to the helix}\index{Osculating circle to the helix}

Our aim here is to produce the geometric construction shown in
Figures~\ref{fig:5:helix} and~\ref{fig:5:assembly}, where a helix
curve is displayed together with the curve of its centers of
curvature (also a helix) and with the osculating circle at a point.

\begin{figure}[h]
\centering\epsfig{file=\figsdir{cap05}spirale.eps,width=0.8\linewidth}
\caption{Views of an ``artistic" assembly including a helix, the
locus of the centers of osculating circles and the osculating circle
at a point
\label{fig:5:helix}}
\end{figure}

So, we start by computing the tangent vector field
\def\aa{{s\over\sqrt{a^2 + b^2}}}
\begin{eqnarray*}
\v{t}(s) = D\v{\alpha}(s) = {1\over\sqrt{a^2 + b^2}}
\vet{-a\sin\aa & a\cos\aa &b }
\end{eqnarray*}
which is a unit vector, and the second derivative
\begin{eqnarray*}
D\v{t}(s) = \kappa(s)\,\v{n}(s) = {1\over{a^2 + b^2}} \vet{-a\cos\aa
& -a\sin\aa & 0 }
\end{eqnarray*}
so that we get
\[
\kappa(s) = {a\over{a^2 + b^2}}
\]
\[
\v{n}(s) = \vet{-\cos\aa & -\sin\aa  & 0 }
\]
Let us note that the helix is a curve of constant curvature.

If $a\not=0$ and $b=0$, then the curve is a circle in $xy$ plane and
$\kappa=1/a$. In other words, the curvature is the reciprocal of 
radius.

If $a=0$ and $b\not=0$, then the curve is a segment of straigth line,
with $\kappa(s)=0$ and $\v{n}(s)$ undefined.


% Eq.~parametriche della tangente
% \def\aa{{s\over\sqrt{a^2 + b^2}}}
% \begin{eqnarray*}
% \v{t}(s) = \v{\alpha}$'$(s) = = {1\over\sqrt{a^2 + b^2}} \vet{-a\sin\aa
% & a\cos\aa &b }
% \end{eqnarray*}

\paragraph{Implementation}\index{Implementation!helix}

We start by implementing, in Script~\ref{script:5:helix:arc}, the
\texttt{helix} vector function $\v{\alpha}:\R\rightarrow\R^{3}$ and the
intrinsic triplet of orthonormal vector functions $\v{t}$, $\v{n}$ and
$\v{b}$. As always, we give a direct translation of the variable-free 
formulation of the $\v{\alpha}$ function, with
\[
\v{\alpha} = \vet{ 
\underline{a} \cos \circ {\mbox{id} \over { \mbox{id}^{1\over 2}
\textstyle\circ 
\underline{{a}^{2}+{b}^{2}}}} & 
\underline{a} \sin \circ {\mbox{id} \over { \mbox{id}^{1\over 2}
\textstyle\circ 
\underline{{a}^{2}+{b}^{2}}}} & 
\underline{b} \ {\mbox{id} \over { \mbox{id}^{1\over 2}
\textstyle\circ 
\underline{{a}^{2}+{b}^{2}}}} }
\]

Notice that, for the sake of abstraction and consistency with the contents
of this chapter, we have used the derivation operator \texttt{D} in
the definition of the \texttt{tangent} and \texttt{normal} functions,
instead using the symbolic expressions previously given.

\begin{script}[Helix parametrized by arc length]
\begin{smallplasm}
DEF helix (a,b::IsReal) = <K:a * cos $\sim$ u, K:a * sin $\sim$ u, K:b * u>\\
WHERE \+\\
  u = ID / SQRT $\sim$ K:(a*a + b*b)\-\\
END;\\[0.3cm]

DEF helixtangent (a,b::IsReal) = AA:D:(helix:<a,b>);\\
DEF helixnormal (a,b::IsReal) = \+\\
    K:((a*a + b*b)/a) scalarVectProd AA:(D $\sim$ D):(helix:<a,b>);\-\\
DEF helixbinormal (a,b::IsReal) = helixtangent:<a,b> VectProd helixnormal:<a,b>;
\end{smallplasm}
\label{script:5:helix:arc}
\end{script}


Notice also that in Script~\ref{script:5:helix:arc} we have redefined
some function identifier already used with a more general meaning. 
So, in the remainder of this chapter the functions called
\texttt{tangent}, \texttt{normal} and \texttt{binormal} uniquely refer
to the unit vector triplet intrinsically related to the \texttt{helix} curve
parametrized by the arc length.

In Script~\ref{script:5:OsculCircle} it is first of all defined the
\texttt{CurvatureCenter}  $\R\rightarrow\R^{3}$ curve, generated as
the vector sum of the \texttt{helix} $\R\rightarrow\R^{3}$ curve
plus the \texttt{normal} $\R\rightarrow\R^{3}$ curve times the
\texttt{K:((sqr:a + sqr:b)/a)} function $\R\rightarrow\R$. Actually the 
implementation produces a whole family of such curves, parametrized 
by the pair of real parameters $a,b$, corresponding respectively to the helix 
radius and to the $z$ interval between two helix points at unit 
curvilinear distance.

In the same script the \texttt{OsculCircle} function is also given, which 
generates the osculating circle at curvilinear ascissa $s$ for the 
curve \texttt{helix:<a,b>}. The image of such function in a concrete 
case is shown in Figure~\ref{fig:5:helix}.

\begin{script}[Osculating circle]
\begin{smallplasm}
DEF helixcurvatureCenter (a,b::IsReal) = \+\\
  helix:<a,b> vectSum (K:((sqr:a + sqr:b)/a) scalarVectProd helixnormal:<a,b>) 
\-\\[0.3cm]

DEF OsculCircle (a,b,s::IsReal) = \+\\
  (T:<1,2,3>:center $\sim$ Rotn:<angle, axis>):circle\-\\
WHERE\+\\
  circle = MAP:((circle3D $\sim$ radius):<a,b, s>):(interval:<0,2*PI>:24),\\
  angle  = (- $\sim$ ACOS $\sim$ InnerProd):<ortho,<0,0,1>>,\\
  axis   = CONS:(helixnormal:<a,b>):s,\\
  center = CONS:(helixCurvatureCenter:<a,b>):s,\\
  ortho ~= CONS:(helixbinormal:<a,b>):s\-\\
END;
\end{smallplasm}
\label{script:5:OsculCircle}
\end{script}

Some utility functions needed to specify the geometric assembly of
Figure~\ref{fig:5:helix} are given in
Script~\ref{script:5:CurveSample}. These include:
\begin{enumerate}
    
\item a function \texttt{circle3D} to produce a circle of radius
\texttt{r} embedded in the coordinate subspace $z=0$ and centered at
the origin; 

\item a function \texttt{radius} to compute the numeric value of the
radius of the osculating circle at curvilinear abscissa $s$.  Such
a number is computed as the norm of the vector difference of two
corresponding points on the \texttt{helix:<a,b>} and
\texttt{curvatureCenter:<a,b>} curves;

\item a function \texttt{CurveGraph} to generate a simplicial
approximation with 90 line segments of the image \texttt{f$[0,6\pi]$}
of its vector function argument \texttt{f};

\end{enumerate}

\begin{script}[Utility functions]
\begin{smallplasm}
DEF circle3D (r::IsReal) = [K:r * COS, K:r * SIN, K:0] $\sim$ s1;\\
DEF radius (a,b,s::IsReal) = (VectNorm $\sim$ VectDiff):\+\\
  < CONS:(helix:<a,b>):s, CONS:(CurvatureCenter:<a,b>):s >;\-\\
DEF CurveGraph (f::IsSeqOf:IsFun) = MAP:(CONS:f  $\sim$ s1):(interval:<0,6*PI>:90);
\end{smallplasm}
\label{script:5:CurveSample}
\end{script}

Finally, in Script~\ref{script:5:assembly} we give the definition of
an \texttt{assembly} object, whose geometric value is shown in
Figure~\ref{fig:5:assembly}.  Such an ``artistic" assembly contains
simplicial approximations of the sets \texttt{helix:<1,2>$[0,2\pi]$}
and \texttt{curvatureCenter:<1,2>$[0,2\pi]$} and a ``mat" (i.e.~2D)
version of the osculating circle at $s=2\pi$, as produced by the
\texttt{JOIN} primitive.

\begin{script}[Artistic assembly]
\begin{smallplasm}
DEF assembly = STRUCT:<\+\\
  (CurveGraph $\sim$ helix):<1,2>,\\
  (CurveGraph $\sim$ curvatureCenter):<1,2>,\\
  (JOIN $\sim$ OsculCircle):<1,2, 2*PI>\-\\
>;\\[0.3cm]

(STRUCT $\sim$ [EMBED:1 $\sim$ BOX:<1,2>, ID]):assembly;
\end{smallplasm}
\label{script:5:assembly}
\end{script}

\begin{figure}[h]
\centering\epsfig{file=\figsdir{cap05}assembly.eps,width=0.5\linewidth}
\caption{``Artistic" assembly of: (i) a \texttt{helix} curve (ii) 
the locus of the centers of its
osculating circles (iii) the osculating circle at $s=2\pi$
\label{fig:5:assembly}}
\end{figure}



% \begin{figure}[h]
% \centering\epsfig{file=\figsdir{cap05}spirale.eps,width=0.7\linewidth}
% \caption{aaa}
% \end{figure}


% \begin{figure}[h]
% %\def\fig:tangente{\scaledpicture 108mm by 62mm (tangente.eps scaled 900)}
% %{\fig:tangente}
% \centering\epsfig{file=\figsdir{cap05}tangente.eps,width=90mm}
% \caption{aaa}
% \end{figure}

\section{Intrinsic properties of a surface}\index{Intrinsic!properties of a surface}

According to the definition of curves, a surface can be defined as a
\emph{point-valued} function $\p{s}$ of \emph{two} real variables.  So, a
\emph{surface} in $ \E^d$ is generated as a set of points by summing to
the origin a vector-valued function $\v{\beta}: \R^{2} \rightarrow
\R^{d}$ of two real variables:
\[
\p{s}(u,v) = \p{o} + \v{\beta}(u,v), \qquad (u,v)\in[0,1]^{2}\subset\R^{2}.
\]
The \emph{image} of the surface is the set $\p{s}[0,1]^{2}$ of its $
\E^d$ points.  The \emph{domain} of the surface is the parameter
interval~$[0,1]^{2}$.

A surface is said to be \emph{regular} where the partial derivatives 
\[
\p{s}_{1}(u,v) = {\partial\p{s}\over\partial u}(u,v)
\quad\mbox{and}\quad
\p{s}_{2}(u,v) = {\partial\p{s}\over\partial u}(u,v)
\]
are linearly independent. In the same way, it is regular (in $\E^{3}$) where 
\[
\p{s}_{1}(u,v) \times \p{s}_{2}(u,v) \not= \v{0}.
\]

In this case, $\p{s}_{1}(u,v)$ and $\p{s}_{2}(u,v)$ give a basis for
the tangent space at $\p{s}(u,v)$.

Hence the tangent vector $\p{c}'(t)$ to a curve $\p{c}(t) =
\p{s}(u(t), v(t))$ on the surface $\p{s}$ can be expressed as a linear
combination of the partial derivatives of $\p{s}$:
\[
\p{c}' = 
{du\over dt}{\partial \p{s} \over \partial u} +
{dv\over dt}{\partial \p{s} \over \partial v} = 
u' \p{s}_{1} + v' \p{s}_{2}.
\] 

\paragraph{Normal to a 3D surface}\index{Normal!to a 3D surface}

The \emph{normal} field $\v{n}: \R^{2}\rightarrow\R^{3}$ to a surface
$\p{s}: \R^{2}\rightarrow\E^{3}$ is given, where the surface is
regular, by the normalized vector product of the partial derivatives:
\[
\v{n} = {\p{s}_{1} \times \p{s}_{2}
\over \module{\p{s}_{1} \times \p{s}_{2}}}
\]

\paragraph{Implementation}\index{Implementation!partial derivatives and normal}

In Script~\ref{script:5:xf} we give the function \texttt{DS} that,
when applied to an integer \texttt{i} and to a surface map
\texttt{surf}, defined as a \emph{sequence} of coordinate functions
$\R^{2}\rightarrow \R$, returns the partial derivative field
\texttt{surf}$_{i} : \R^{2}\rightarrow \E^{3}$ ($1\leq i\leq 2$). 
This result is obtained by suitably applying to each component of
\texttt{surfMap} the \texttt{Dp} linear operator given in
Script~\ref{script:5:Dp}.

Also, a \emph{normal field} operator \texttt{N} is defined, as the
normalized vector product of the (tangent) fields generators
\texttt{DS:1} and \texttt{DS:2}.  The \texttt{vectorProd} operator can
be found in Script~\ref{script:vectorProd}.  Clearly, specific tangent
or normal fields are returned when such operators are applied to a
specific surface mapping.

\begin{script}[Partial derivatives and normal]
\begin{smallplasm}
DEF X(i::isIntPos)(f::IsFun)(a::IsPoint) = Dp:i:f:a:<1>;\\
DEF Norm3 (x,y,z::IsFun) = < x/den, y/den, z/den >\\
WHERE\+\\
  den = SQRT  $\sim$  +  $\sim$  AA:sqr  $\sim$  [x, y, z],\\
  sqr = ID * ID\-\\
END;\\[0.3cm]

DEF DS \index{Library functions!Derivatives library!{\tt  Ds}} (i::IsIntPos)(surfMap::IsSeqOf:IsFun) = AA:(X:i): surfMap;\\
DEF N \index{Library functions!Derivatives library!{\tt  N}} = Norm3  $\sim$  VectProd  $\sim$  [DS:1, DS:2];
\end{smallplasm}
\label{script:5:xf}
\end{script}

\begin{example}[Graph of partial derivatives]
    
We give in Script~\ref{script:5:nmap} an implementation of the
geometric construction shown in Figure~\ref{fig:5:normal}, where a
graph is shown of both the tangent vectors
\texttt{DS:1}$(\alpha)({\pi\over 5}, {\pi\over 10})$ and
\texttt{DS:2}$(\alpha)({\pi\over 5}, {\pi\over 10})$, and of the
normal vector \texttt{N}$(\alpha)({\pi\over 5}, {\pi\over 10})$, for a
surface $\alpha = (u, v, \sin\, u\sin\, v)$.  in Figure~\ref{fig:5:normal} the
two curves $\alpha(u,{\pi\over 10})$ and $\alpha({\pi\over 5}, v)$ are
also shown.  

This geometric construction is produced by exporting the geometric
value of the \texttt{graph} object in Script~\ref{script:5:nmap}, for
example as:
\begin{smallplasm}
VRML:graph:$'$out.wrl$'$;
\end{smallplasm}


\begin{figure}[h]
\centering\epsfig{file=\figsdir{cap05}normal.eps,width=0.5\linewidth}
\caption{First partial derivatives and normal vector in $\alpha({\pi\over
5}, {\pi\over 10})$ to a surface $\alpha(u,v)=(u,v,\sin u\, \sin v)$
\label{fig:5:normal}}
\end{figure}


Notice that the variable-free representation (see
Section~\ref{sec:5:free}) of the studied surface is
\[
\alpha := ( \sigma(1), \sigma(2), 
(\sin\circ\,\sigma(1))(\sin\circ\,\sigma(2)) )
\]
which is directly translated into the \texttt{surf} sequence.  The
functions \texttt{interval} and \texttt{interval2D}, generating suitable
decompositions of a 1D and 2D interval, respectively, are given in
Script~\ref{script:5:toolbox}.

\begin{script}[Graph of partial derivatives]
\begin{smallplasm}
DEF surf = < S1, S2, SIN $\sim$ S1 * SIN $\sim$ S2 >;\\
DEF domain  = interval2D:<0,0,PI,PI>:<10,10>;\\[0.3cm]

DEF graph = STRUCT:< \+\\
  MAP:surf:domain CREASE (PI/2),\\
  MAP:surf:((T:2:(PI/10)  $\sim$  EMBED:1  $\sim$  interval:<0,PI>):10),\\
  MAP:surf:((T:1:(PI/5)  $\sim$  R:<1,2>:(PI/2)  $\sim$  EMBED:1  $\sim$  
  interval:<0,PI>):10),\\
  T:<1,2,3>:p,\\
  Segment:1:< <0,0,0>, CONS:(DS:1:surf):<PI/5, PI/10> >,\\
  Segment:1:< <0,0,0>, CONS:(DS:2:surf):<PI/5, PI/10> >,\\
  Segment:1:< <0,0,0>, CONS:(N:surf):<PI/5, PI/10> >\-\\
> \\
WHERE p = CONS:surf:<PI/5, PI/10> END;
\end{smallplasm}
\label{script:5:nmap}
\end{script}

\label{ex:5:nmap}
\end{example}


\begin{example}[Sampling of normal field]
    
A graphical representation of the normal field $\v{n}$ is produced by
the \texttt{N\_Map} function of Script~\ref{script:5:nfield}, and
displayed in Figure~\ref{fig:5:nfield} for the normal field induced by
the $\alpha(u,v)=(u,v,\sin u\, \sin v)$ surface, with
$(u,v)\in[0,\pi]^{2}$.  

In particular, Figure~\ref{fig:5:nfield} is generated by graphically
browsing the geometric value produced by the \pl\ interpreter when
evaluating the last expression of Script~\ref{script:5:nfield}.

\begin{script}[Sampling of normal field]
\begin{smallplasm}
DEF N\_Map (f::IsSeqOf:IsFun; dom::IsPol) = (STRUCT \+\\
    $\sim$  AA:(Segment:0.35  $\sim$  [CONS:f, vectSum  $\sim$  [CONS:f,(CONS $\sim$ N):f] ])\\
    $\sim$  S1  $\sim$  UKPOL):dom;\-\\[0.3cm]

N\_Map:< surf, domain > STRUCT (MAP:surf:domain CREASE (PI/2))
\end{smallplasm}
\label{script:5:nfield}
\end{script}

The reader should notice that the \texttt{N\_Map} function given here
is very similar to the \texttt{GradMap} function of
Script~\ref{script:5:GradMap}. 

\begin{figure}[h]
\centering\epsfig{file=\figsdir{cap05}nfield.eps,width=0.5\linewidth}
\caption{Sampling of the (scaled) normal field $\v{n}(u,v)$ to a surface 
$\alpha(u,v)=(u,v,\sin u\, \sin v)$, over the domain $[0,\pi]^{2}$
\label{fig:5:nfield}}
\end{figure}

\label{ex:5:nfield}
\end{example}


\subsection{First fundamental form}\index{First!fundamental form}

Let us consider a curve $\p{c}(t) = \p{s}(u(t), v(t))$ on the surface 
$\p{s}(u,v)$, with $a\leq t\leq b$, and let $s$ be the arc length:
\[
s(t) = \int_{a}^{t} ds = \int_{a}^{t} \module{\p{c}'(t)} dt.
\]

Hence we have:
\begin{eqnarray*}
\left({ds \over dt}\right)^{2} &=& \module{\p{c}'}^{2} %\\
 = \p{c}'\cdot\p{c}'%\\
= (u' \p{s}_{1} + v' \p{s}_{2}) \cdot (u' \p{s}_{1} + v' \p{s}_{2}) \\
 &=& (\p{s}_{1}\cdot\p{s}_{1}) u' + 2 (\p{s}_{1}\cdot\p{s}_{2}) u' v'
      + (\p{s}_{2}\cdot\p{s}_{2}) v' 
\end{eqnarray*}
from where, by using the Gauss  notation:
\begin{equation}
\p{s}_{1}\cdot\p{s}_{1} = \mbox{E}, \qquad
\p{s}_{1}\cdot\p{s}_{2} = \mbox{F}, \qquad
\p{s}_{2}\cdot\p{s}_{2} = \mbox{G} .
\label{eq:5:firstform}
\end{equation}
we have
\[  
\left({ds \over dt}\right)^{2}
= \mbox{E} u' + 2 \mbox{F} u' v' + \mbox{G} v',
\]
and remembering that $u' = {du \over dt}$ and  $v' = {dv \over dt}$, we get 
the intrinsic quantity called \emph{First Fundamental Form} or also 
\emph{Metric Form}
\[
ds = \mbox{E}\, du + 2\, \mbox{F}\, du\, dv + \mbox{G}\, dv,
\]
in differential notation, which arises in computing the arc length of
curves on surfaces, as well as the angle between tangent vectors on a
surface.

In fact, let $\v{v} = a \p{s}_{1} + b \p{s}_{2}$ and $\v{w} = c
\p{s}_{1} + d \p{s}_{2}$ be tangent vectors to a surface $\p{s}$
in the same point. We can write
\begin{eqnarray*}
\p{v} \cdot \p{w} &=& (a \p{s}_{1} + b \p{s}_{2}) \cdot (c \p{s}_{1} +
d \p{s}_{2})\\
&=& ac\, (\p{s}_{1}\cdot\p{s}_{1}) + (ad + bc)\, (\p{s}_{1}\cdot\p{s}_{2}) +
bd\, (\p{s}_{2}\cdot\p{s}_{2})\\
&=& ac\, \mbox{E} + (ad + bc)\, \mbox{F} + bd\, \mbox{G}\\
&=& \mat{a & b}\mat{ \mbox{E} &  \mbox{F} \\ \mbox{F} &  
\mbox{G}}\mat{c \\ d}.
\end{eqnarray*}

The matrix in the above formula is called \emph{matrix of the first
fundamental form} or also \emph{metric tensor}. 

Relabeling $\p{s}_{i}\cdot\p{s}_{j} = g_{ij}$, it is possible to write 
for the differential area element:    
\[
\mat{ E & F \\ F & G } = 
\mat{ g_{11} & g_{12} \\ g_{21} & g_{22} } 
\]
and, by labeling $du = du^{1}$ and $dv = dv^{2}$
\[
ds^{2} = \sum_{i,j} g_{ij}\, du^{i}\, du^{j}, \qquad i,j\in \{1,2\}.
\]

Another important property of the metric tensor is:
\begin{eqnarray*}
\module{ \p{s}_{1} \times \p{s}_{2} }^{2}
&=& (\p{s}_{1}\cdot\p{s}_{1})(\p{s}_{2}\cdot\p{s}_{2}) - 
 (\p{s}_{1}\cdot\p{s}_{2})^{2}\\
&=& \mbox{E}\,\mbox{G} - \mbox{F}^{2}\\
&=& \det \mat{ g_{11} & g_{12} \\ g_{21} & g_{22} }\\
&=:& g
\end{eqnarray*}

\begin{example}
    
Let us try the forms $E$, $F$ and $G$ on the $(0,0)$ corner of the
$[0,1]^{2}$ domain of the $\v{s} = (u, v, \sin u\,\sin v)$ surface
map $[0,1]^{2}\rightarrow \E^{3}$ given in Script~\ref{script:5:nmap}.  

In this case we have 
\begin{eqnarray*}
\p{s}_{1}(u,v)&=&(1,0,\cos u\,\sin v)
\\
\p{s}_{2}(u,v)&=&(0,1,\sin u\,\cos v),
\end{eqnarray*}
so that
\begin{eqnarray*}
E(\p{s}) (0,0) &=& (\p{s}_{1}\cdot\p{s}_{1})(0,0) 
= ((1,0,\cos u\,\sin v)\cdot(1,0,\cos u\,\sin v))(0,0) = 1\\
F(\p{s}) (0,0) &=& (\p{s}_{1}\cdot\p{s}_{2})(0,0) = ((1,0,\cos u\,\sin v)
\cdot (0,1,\sin u\,\cos v))(0,0) = 0\\
G(\p{s}) (0,0) &=& (\p{s}_{2}\cdot\p{s}_{2})(0,0) 
= ((0,1,\sin u\,\cos v)\cdot(0,1,\sin u\,\cos v))(0,0) = 1
\end{eqnarray*}

\end{example}
    
\paragraph{Implementation}\index{Implementation!first fundamental form}

The functional elements $E$, $F$ and $G$ of the metric tensor are
easily computed in \pl\ as the functions \texttt{E1}, \texttt{F1} and
\texttt{G1} of Script~\ref{script:5:form1}.  According to their
defining formulas~(\ref{eq:5:firstform}), such functions are
implemented as the composition of the \texttt{innerprod} operator
(given in Script~\ref{script:UnitVect}) with the appropriate
\texttt{CONS} of partial derivation operators \texttt{DS:$i$} given in
Script~\ref{script:5:xf}.

\begin{script}[First fundamental form]
\begin{smallplasm}
DEF E1 = innerprod $\sim$ [DS:1, DS:1];\\
DEF F1 = innerprod $\sim$ [DS:1, DS:2];\\
DEF G1 = innerprod $\sim$ [DS:2, DS:2];\\[0.3cm]

E1:surf:<0,0> $\equiv$ 1.0\\
F1:surf:<0,0> $\equiv$ 0.0\\
G1:surf:<0,0> $\equiv$ 1.0
\end{smallplasm}
\label{script:5:form1}
\end{script}



\subsection{Second fundamental form}\index{Second fundamental form}

The tensor of the \emph{second fundamental form} is defined by the 
matrix:
\[
\mat{
L & M\\M & N
}
\]
where the linear forms $L$, $M$ and $N$ of the tensor are respectively
defined as:
\begin{eqnarray*}
L &=& \v{n}\cdot {\partial^{2} \over{ \partial u}^{2}}\\
M &=& \v{n}\cdot {\partial^{2} \over{ \partial u}{ \partial v}}\\
N &=& \v{n}\cdot {\partial^{2} \over{ \partial v}^{2}}
\end{eqnarray*}

\begin{example}
Let us compute the values of
the $L$, $M$ and $N$ forms on the $(0,0)$ corner of the $[0,1]^{2}$
domain of the $\v{s}=(u,v,\sin u\,\sin v)$ surface.  

In this case we have
\begin{eqnarray*}
\v{n} &=& \v{s}_{1}\times\v{s}_{2} \\
 &=& (1,0,\cos u\,\sin v) \times (0,1,\sin u\,\cos v) \\
 &=& (- \cos u\,\sin v, \sin u\,\cos v, 1)
\end{eqnarray*}
and
\begin{eqnarray*}
\v{s}_{11} &=& {\partial^{2} \v{s} \over{ \partial u}^{2}} 
\ =\ (0,0,-\sin u\,\sin v)\\
\v{s}_{12} &=& {\partial^{2} \v{s} \over{ \partial u}{ \partial v}} 
\ =\ (0,0,\cos u\,\cos v)\\
\v{s}_{22} &=& {\partial^{2} \v{s} \over{ \partial v}^{2}} 
\ =\ (0,0,-\sin u\,\sin v)
\end{eqnarray*}
so that
\begin{eqnarray*}
L(\p{s}) (0,0) &=& (\p{n}\cdot\p{s}_{11})(0,0) \\
&=& ((- \cos u\,\sin v, \sin u\,\cos v, 1)\cdot(0,0,-\sin u\,\sin v))(0,0) = 0\\
M(\p{s}) (0,0) &=& (\p{n}\cdot\p{s}_{12})(0,0)\\
&=& ((- \cos u\,\sin v,
\sin u\,\cos v, 1) \cdot (0,0,\cos u\,\cos v))(0,0) = 1\\
N(\p{s}) (0,0) &=& (\p{n}\cdot\p{s}_{22})(0,0) \\
&=& ((- \cos u\,\sin v, \sin u\,\cos v, 1)\cdot(0,0,-\sin u\,\sin v))(0,0) = 0
\end{eqnarray*}

\end{example}

\paragraph{Implementation}\index{Implementation!second fundamental form}

In accordance with the implementation of the functional elements of
the metric tensor, we give in Script~\ref{script:5:form2} an
implementation of the elements $L$, $M$ and $N$ of the second
fundamental form, respectively denoted as \texttt{L2}, \texttt{N2} and
\texttt{M2} in the implementation.

\begin{script}[Second fundamental form]
\begin{smallplasm}
DEF L2 = innerprod $\sim$ [N, DS:1  $\sim$  DS:1];\\
DEF M2 = innerprod $\sim$ [N, DS:1  $\sim$  DS:2];\\
DEF N2 = innerprod $\sim$ [N, DS:2  $\sim$  DS:2];\\[0.3cm]

L2:surf:<0,0> $\equiv$ 0.0\\
M2:surf:<0,0> $\equiv$ 1.0\\
N2:surf:<0,0> $\equiv$ 0.0
\end{smallplasm}
\label{script:5:form2}
\end{script}
A computational check was also executed, by computing 
$L(\p{s}) (0,0)$, $M(\p{s}) (0,0)$ and $N(\p{s}) (0,0)$, and we found 
numerically the results  already symbolically obtained in the above 
paragraph.



\subsection{Gauss curvature}\index{Gauss curvature}

Let consider the normal $\v{n}(u,v)$ to a point $\v{s}(u,v)$ of a
surface $S = \v{s}[0,1]^{2}$ in $\E^{3}$ and a unit tangent vector
$\v{v}$ to such a surface in a point $\v{s}(u,v)$.  Then consider the
curve $\v{\alpha}_{\v{v}}$ generated by sectioning the surface with
the normal plane passing for $\v{n}$ and $\v{v}$.  The curvature of
the osculating circle to $\v{\alpha}_{v}$ is called the \emph{normal
curvature of $S$ at $\v{s}(u,v)$ in the $\v{v}$ direction}, and is
denoted as $k_{n}(\v{v})$.  This function varies between maximum
and minimum values, which are reached at two orthogonal directions,
called \emph{principal directions}.  The maximum and minimum normal
curvatures at a point are called \emph{principal curvatures} $k_{1}$
and $k_{2}$ at that point.

The product of the principal curvatures is called the \emph{Gauss 
curvature} $K := k_{1}k_{2}$.

The  \emph{Gauss curvature field} on a surface $\p{s}:[0,1]^{2}\rightarrow
S\subset\E^{3}$ can be defined as the ratio of determinants of the
matrices of the second and first fundamental forms:
\[
K(\p{s}) : S \rightarrow\R : {L(\p{s})N(\p{s}) - M^{2}(\p{s}) 
\over E(\p{s})G(\p{s}) - F^{2}(\p{s})}
\]

\paragraph{Implementation}\index{Implementation!Gauss curvature}

We already used the symbol \texttt{N} to denote the normal field to a
surface.  Hence in Script~\ref{script:5:form2}, we used the symbols
\texttt{L2}, \texttt{M2} and \texttt{N2} to denote the elements of the
second fundamental form. In Script~\ref{script:5:gausscurvature} we 
give the 
\[
\texttt{GaussCurvature} : ([0,1]^{2}\rightarrow \E^{3}) \rightarrow 
(\E^{3}\rightarrow \R)
\]
field operator which, when applied to a surface mapping, returns its
curvature field.  For the sake of readibility, we would like to write the
\texttt{GaussCurvature} function using only infix operators, as,
e.g., in
\begin{smallplasm}
    ((L * N) - (M * M)) / ((E * G) - (F * F))
\end{smallplasm}
but this is not actually possible with the current \pl\
implementation, since the algebraic operators would be ``raised" too
many times.  Hence, the standard \fl\ prefix and \texttt{CONS}ed form
must instead be used.

\begin{script}[Gauss curvature]
\begin{smallplasm}
DEF GaussCurvature \index{Library functions!Derivatives library!{\tt  Gausscurvature}} = \+\\
/ $\sim$ [-$\sim$[*$\sim$[L2,N2], 
*$\sim$[M2,M2]], -$\sim$[*$\sim$[E1,G1],*$\sim$[F1,F1]]];
\-\\[0.3cm]

GaussCurvature:surf:<0,0> $\equiv$ -0.9999999999333331\\
GaussCurvature:surf:<PI/2,PI/2> $\equiv$ 1.000000165454544
\end{smallplasm}
\label{script:5:gausscurvature}
\end{script}
Clearly, the given computational examples are affected by
approximation and round-off errors, as should be clear by remembering
that we are using numeric derivation methods.  Once again, the
$\texttt{surf} = (u, v, \sin u\,\sin v)$ mapping, used for the two
checks above, is the one given in Script~\ref{script:5:nmap}.


\section{Examples}

An example of computation and presentation of the Gauss curvature
field on a surface, generated by interpolation of two extreme curves
with assigned extreme derivative fields, is given in this section. 
The surface generation by ``transfinite" interpolation of curve maps,
sometimes called \emph{skinning} in CAD systems, is discussed in
Chapter~\ref{sect:12:transfinite}, but is anticipated here to work out
a realistic example.

\subsection{Color map of the Gauss  curvature field}\index{Color!map of Gauss  curvature field}
\label{subsec:5:gaussmap}


A very interesting application of many concepts studied in this book
is discussed here.  It allows experimentation with several features of
the \pl\ language at the same time.  In particular:
\begin{enumerate}
    
    \item A smooth surface $\p{s}:[0,1]^{2}\rightarrow S$, with
    $S\subset\E^{3}$, is generated by using transfinite interpolation
    methods.  For this purpose two boundary curves
    \begin{eqnarray}
	\p{c}_{1}&:&[0,1]\rightarrow S : u\mapsto \p{s}(u,0)\\
    	\p{c}_{2}&:&[0,1]\rightarrow S : u\mapsto \p{s}(u,1),
    \end{eqnarray}
    to be interpolated by $\p{s}$ are given, and are also assigned two
    fields of first derivatives along the images of such
    curves.
    
    \item The field $K : S\rightarrow\R$ of Gauss  curvature is
    \emph{sampled} at the vertices of a cell decomposition defined
    upon the $\p{s}[0,1]^{2}$ image of the surface.  The result is
    represented as a polyhedral approximation, called
    \texttt{CurvatureField}, of the 2D manifold
    \[
    (\mbox{id},K)(\p{s})[0,1]^{2}
    \]
    embedded in $4$-dimensional space $\E^{3}\times\R$.
    
    \item Finally, the $4$-th coordinate of vertices of this object is
    checked for positivity/negativity and used to label each surface
    vertex with a color triplet in two different ranges of colors, 
    thus
    transforming the 2-dimensional manifold \texttt{CurvatureField} in
    4D space into the 2-dimensional manifold \texttt{colorGauss} in 6D
    space.
    
    \begin{enumerate}
	
	\item 
    At exporting time into a \textsf{VRML}\index{VRML} file, the first three
    coordinates are interpreted as position of points, the last three
    coordinates are interpreted as \emph{color per vertex} (see for
    further details Section~\ref{sec:10:vrml}).
    
    \item 
    The resulting surface, shaded between \emph{cyan} and \emph{white}
    where the Gauss curvature is positive, and between \emph{blue} and
    \emph{cyan} where the curvature is negative, is shown in
    Figure~\ref{fig:5:gauss}.
    \end{enumerate}

\end{enumerate}
Clearly, such a very simple computational solution to a quite complex
scientific visualization problem, was possible because of the
dimension-independent \pl\ approach to geometric modeling.



\paragraph{Surface generation}\index{Surface!generation}

First of all, we generate a surface mapping $\p{s}(u,v)$, with
$(u,v)\in [0,1]^{2}$, by Hermite\footnote{Discussed in Section~\ref{subsec:11:hermite}
for the standard case of interpolation of points and tangent vectors,
and in Section~\ref{sect:12:transfinite} for the transfinite interpolation of curves,
surfaces and higher dimensional manifolds, together with assigned
boundary derivative fields.} interpolation of two boundary curves
\[
\p{s}(u,0) = \texttt{c1}  \quad\mbox{and}\quad \p{s}(u,1) = 
\texttt{c2},
\]
and two boundary fields of
(constant) first derivatives 
\[
\p{s}_{2}(u,0) = (\underline{1},\underline{1},\underline{1})
\quad\mbox{and}\quad \p{s}_{2}(u,1) =
(\underline{-1},\underline{-1},\underline{-1}).
\]

It is not necessary to understand here how the $\p{s}$ mapping is
generated.  The used ``transfinite" interpolation is discussed in
Section~\ref{sect:12:transfinite}.  Anyway, we have:
\[
\p{s}(u,v) = \p{s}(u,0) \v{h}_{0}(v) + \p{s}(u,1) \v{h}_{1}(v) +
\p{s}_{2}(u,0) \v{h}_{2}(v) + \p{s}_{2}(u,1) \v{h}_{3}(v),
\]
where $(\v{h}_{0}(v), \v{h}_{1}(v), \v{h}_{2}(v), \v{h}_{3}(v))^{T}$ is
the Hermite's basis of cubic polynomials.  

\paragraph{Implementation}\index{Implementation!graph of Gauss curvature field}

We are interested here to the \texttt{Surface} mapping, generated by
the \texttt{CubicHermite} operator given in Script~\ref{script:12:transfhermite}.  Analogously,
\texttt{c1} and \texttt{c2} are two cubic Hermite's curves generated
by giving two extreme points and two extreme tangents.  Finally,
\texttt{dom} is a polyhedral complex partitioning the real interval
$[0,1]$ into $12$ segments, and \texttt{CurvatureField} is the
resulting polyhedral approximation of the 2D manifold
$(\mbox{id},K)(\p{s})[0,1]^{2}$ embedded in 4-dimensional space.

\begin{script}[Geometric data]
\begin{smallplasm}
DEF c1 = CubicHermite:S1:<<1,0,0>,<0,1,0>,<0,3,0>,<-3,0,0>>;\\
DEF c2 = CubicHermite:S1:<<0.5,0,0>,<0,0.5,0>,<0,1,0>,<-1,0,0>>;\\
DEF Surface = CubicHermite:S2:<c1,c2,<1,1,1>,<-1,-1,-1>>;\\[0.3cm]

DEF dom = interval:<0,1>:12;\\
DEF CurvatureField = \+\\
  MAP: ((CONS $\sim$ AR $\sim$ [ID, GaussCurvature]):Surface): (dom * 
dom);\-\\[0.3cm]

CurvatureField $\equiv$ A-Polyhedral-Complex\{2,4\}
\end{smallplasm}
\label{script:5:geomdata}
\end{script}


The transformation from the $2$-manifold \texttt{CurvatureField} in 4D
space, encoding the Gauss  curvature as the $4$-th coordinate, to the
$2$-manifold \texttt{colorGauss} in 6D space, encoding such curvature
as a \textsc{rgb} color triplet, is performed in
Script~\ref{script:5:colorGauss}.  For this purpose the maximum and
minimum values of the $4$-th coordinate of \texttt{CurvatureField}
vertices are stored into \texttt{curvmax} and \texttt{curvmin},
respectively, and used to map the curvature values to the appropriate 
triples in the two color ranges used for positive and negative 
curvatures, respectively.

\begin{script}[Color map of curvature]
\begin{smallplasm}
DEF colorGauss \index{Library functions!Derivatives library!{\tt  Colorgauss}} = MAP:(IF:< IsRealPos $\sim$ s4, \+\\{}
     [s1,s2,s3,/ $\sim$ [s4,K:(curvmax - curvmin)],K:1,K:1], \\{}
     [s1,s2,s3,K:0,/ $\sim$ [s4,K:(curvmin - curvmax)],K:1] %\-\\
>):CurvatureField\-\\
WHERE\+\\
  curvmax = MAX:4:CurvatureField, \\
  curvmin = MIN:4:CurvatureField \-\\
END;\\[0.3cm]

VRML:(colorGauss CREASE (PI/2)):$'$out.wrl$'$;\\
colorGauss $\equiv$ A-Polyhedral-Complex\{2,6\}
\end{smallplasm}
\label{script:5:colorGauss}
\end{script}

\begin{figure}[h]
% \begin{minipage}[c]{0.245\linewidth}
%     \centering\epsfig{file=\figsdir{cap05}gauss1.eps,width=\linewidth}
% \end{minipage}
\begin{minipage}[c]{0.327\linewidth}
    \centering\epsfig{file=\figsdir{cap05}gauss2.eps,width=\linewidth}
\end{minipage}
\begin{minipage}[c]{0.327\linewidth}
    \centering\epsfig{file=\figsdir{cap05}gauss6.eps,width=\linewidth}
\end{minipage}
\begin{minipage}[c]{0.327\linewidth}
    \centering\epsfig{file=\figsdir{cap05}gauss5.eps,width=\linewidth}
\end{minipage}
\caption{Gauss  curvature field: (a) surface generated by Hermite's
transfinite interpolation of two curves and two (constant)
vector fields (b) subset of negative Gauss  curvature (c) subset of
positive Gauss  curvature
\label{fig:5:gauss}}
\end{figure}



%%%%%%%%%%%%%%%%%%%%%%
                     
    \end{document}   
                     
%%%%%%%%%%%%%%%%%%%%%%
